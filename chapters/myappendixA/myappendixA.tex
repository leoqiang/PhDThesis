\chapter{Appendix A: expression compendium exploration 
functionalities}\label{ch:apd-colombos}

\ldots

\instructionsappendices


%\label{apd:colombos}

COLOMBOS provides rich functionalities to create and/or edit expression 
'modules'. In chapter \ref{ch:distiller}, we explored the option that let 
COLOMBOS automatically identify relevant condition contrasts based on 
the user specified query gene(s).  Alternatively one can also 
specify the conditions in interests and letting COLOMBOS automatically identify 
the genes that are over-expressed or under-expressed in them. Furthermore, it 
is also possible to maintain the full control by specifying both the genes and 
the conditions in interests in order to check the behavior logged in the 
existing experimental data.

When specifying genes of interests, apart from manually inputting gene 
information, one can also select genes based on other annotations obtained from 
public databases, such as the transcription factor or sigma factor that 
regulates the gene’s transcription (\cite{Gama-Castro2008}), the pathway a gene 
belongs to (\cite{Karp2005}), or the transcription unit a gene belongs to 
(\cite{Karp2005}). 
These functionalities are available as options in the gene selection section 
of the module creation panel. 

Additionally, several alternative functions are provided to choose the 
condition contrasts besides the condition selection based on expression values 
as used here. In case one is interested in specific experiments, one can select 
them directly. When a user is interested in specific types of condition 
properties, a condition hierarchy is available. The properties are grouped into 
4 major categories: Genomic, Growth, Medium, and Treatment. Subcategories exist 
under each major category to further classify them. This functionality can be 
reached through option `By annotation' in the condition selection section 
of the module creation panel. One can also use the condition 
property ontology in a similar way by selecting `By ontology' in the condition 
selection section.

For automatic expression value based contrasts or genes selection procedures, 
the calculations used to score relevance of a contrast for 
a set of genes, similarity of genes across a set of contrasts, or variability 
of a gene across a set of contrasts, are explained in the following sections.  
In addition, we also explained the enrichment calculation, the user module data 
exportation, and the advantages for the registered users here.

Even more options are available when the user modifies an existing 
module. To check all available functionalities, we refer the online manual of 
COLOMBOS.



\section{Contrast relevance score}\label{apd:contrast-score}

The default relevance score $c$ of a condition contrast for a group
of genes is calculated as the absolute inverse coefficient of
variation of those genes’ expression values in this contrast. It
is defined as the absolute mean divided by the standard deviation
of the genes’ expression values:

\begin{equation}
c=\frac{|\mu|}{\sigma}
\end{equation}
 
On the one hand, for expression values of the same mean, the
higher the score, the less sparse the values are. It prioritizes
the contrasts where genes’ expression values are more
consistent. On the other hand, for expression values of the same
standard deviation, the higher the score, the higher the mean. It
prefers the contrasts where genes are highly expressed. The score
thus serves as a measure that values both magnitude of expression
change in response to a condition contrast, as well as coherence
of expression within that contrast. The score identifies the most
relevant contrasts as those where the genes 'act as one', showing
the same, preferentially large, magnitude of expression change
with individual variations ideally only constituting random
Gaussian noise. From this notion, the score represents the number
of standard deviations the mean expression value of this
distribution is situated away from $0$, and can be interpreted as a
$Z$-score for the selected genes' expression change as a
whole. (Note that in case of only one gene, the score of a
condition contrast is degraded to the absolute expression value
of that gene under it.)

We have also provided an alternative measure for the contrast
relevance (selectable by the box in the top right of the contrast
selection window) called 'M value cutoff'.  The score assigned to
the contrasts in this case is the minimum M value (i.e. the
logratio) for the considered genes in case all genes' M values
are positive, or the maximum absolute M value in case all genes'
M values are negative.  In case of both positive and negative M
values exist for the considered genes, the contrast gets a score
of 0.



\section{Gene similarity score}\label{apd:gene-score}

The default similarity between a gene and a module's mean
profile, is the \textit{Uncentered Pearson’s correlation} calculated based
on the formula:

\begin{equation}
r_v=\frac{1}{n}\sum_{i-1}^n(\dfrac{x_i}{\sigma_x^{(0)}})(\dfrac{y_i}{\sigma_y^{(0)}})
\end{equation}

% NOTE: using extra tags to control the math displayed inline with the text
% Ref: 
%http://tex.stackexchange.com/questions/32824/show-inline-math-as-if-it-were-display-math
where $\textstyle\sigma_x^{(0)}=\sqrt{\frac{1}{n}\sum\limits_{i}x_i^2}$; and 
$\textstyle\sigma_y^{(0)}=\sqrt{\frac{1}{n}\sum\limits_{i}y_i^2}$
Here $x_i$ represents the candidate gene expression data at
condition contrast $i$, whereas $y_i$ represents a module’s mean
expression value at contrast $i$. $\sigma_x^{(0)}$ and
$\sigma_y^{(0)}$ are both uncentered standard deviations assuming
zero mean of the population, hence they are marked with
superscript ‘$(0)$’. The higher the $v_r$, the more similar the
expression profile of a gene is to a module’s mean expression
profile.

We have also provided an alternative measure for the gene
similarity (selectable by the box in the top right of the
contrast selection window) that calculates an uncentered version
of the \textit{Spearman rank correlation}. The Spearman rank correlation
is calculated in the same way as the Pearson correlation but on
the ranks of the data instead of the data itself. To calculate an
uncentered version, instead of ranking all values from low to
high, the positive logratios are ranked from low to high while
the negative logratios are ranked from high to low and then
assigned a negative sign; the mean rank is assumed $0$.  This
uncentered Spearman rank correlation, compared to the uncentered
Pearson correlation, has the advantage of being able to capture
non-linear similarities, but the disadvantage of ignoring the
actual magnitudes of expression changes and their distributions.

For ranking genes, there are three options provided based on the
uncentered (rank) correlation score calculated. First one is
‘positive’ which uses $v_r$ directly as final score.  The second
option ‘absolute’ takes $|v_r|$ . It ranks both correlated and
anti-correlated genes based solely on their similarities. Instead, 
the third option ‘negative’ takes $-v_r$ as a score to favor only 
the anti-correlated genes.



\section{Gene variability}\label{apd:gene-sd}

The variability of a gene's expression value $x$ for conditions $i=1,…,n$ is 
calculated as the uncentered standard deviation: 

\begin{equation}
\sigma_x^{(0)}=\sqrt{\frac{1}{n}\sum\limits_{i=1}^{n}x_i^2}
\end{equation}


\section{Enrichment calculation}\label{apd:enrichment}

The GO (Gene Ontology) enrichment $p$-value is calculated based on a 
\textbf{hypergeometric distribution}. The lower the $p$-value, the lower the 
chance of a GO term appearing in a random gene set having the same number of 
genes as the module in comparison, and the more significant a GO term is 
enriched in the module.
\todo{[apdxA]Add formula and brief explanations here}

\section{User module data export}\label{apd:data-download}

Once a module is created, it can be exported in the format described 
in Section \ref{sec:dist-format-col}. 
To do so first click the module in the workspace to show its 
overview tab. 
Then click the `download' button located at the up right corner of 
this tab to show the download window. 
This automatically sends a request to the server to generate the data file for 
the current module. 
When ready, a link to the generated data file will appear in the window. 
User can click it to save the module data.


\section{Registered users}\label{apd:colombos-user}\todo{[apdxA]extend and 
update 
it}

An anonymous user can use the analysis interface without logging in, but the 
data generated will be lost when the webpage is closed. A registered user has 
the advantage of being able to save the workspace, and load it again any other 
time when logged in. It takes only two steps to create a new user. After 
logging in, the `Data' menu in the workspace panel will have items 
`Load', `Save', and `Save As' enabled.

\todo{[apdxA]user information and workspace management panel}



%%%%%%%%%%%%%%%%%%%%%%%%%%%%%%%%%%%%%%%%%%%%%%%%%%
% Keep the following \cleardoublepage at the end of this file, 
% otherwise \includeonly includes empty pages.
\cleardoublepage

% vim: tw=70 nocindent expandtab foldmethod=marker foldmarker={{{}{,}{}}}
