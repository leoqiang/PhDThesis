\chapter{Appendix A: expression exploration functionalities}\label{ch:myappendix}

\ldots

\instructionsappendices


\section{Appendix A: expression exploration functionalities}

COLOMBOS provides rich functionalities to  create and/or edit expression 'modules', 
some of which are based on the expression values themselves.  The calculations used in 
these procedures to score relevance of a contrast for a set of genes, similarity of genes 
across a set of contrasts, or variability of a gene across a set of contrasts, are explained in this supplementary.  


\subsection{Contrast relevance score}
The  relevance  score  c  of a condition contrast for a group of genes is calculated as the 
inverse coefficient of variation of those genes’ expression values in this contrast. It is 
defined as the mean divided by the standard deviation of the genes’ expression values: 


 
On the one hand, for expression values of the same mean, the higher the score, the less 
sparse the values are. It prioritizes the contrasts where genes’ expression values are more 
consistent. On the other hand, for expression values of the same standard deviation, the 
higher the score, the higher the mean. It prefers the contrasts where genes are highly 
expressed. The score thus serves as a measure that values both magnitude of expression 
change in response to a condition contrast, as well as coherence of expression within that 
contrast.  

In case of one gene, the score of a condition contrast is then degraded to the absolute 
expression value of that gene under it.  


\subsection{Gene similarity score}

The  similarity between gene profiles  is the Uncentered Pearson’s correlation calculated 
based on the formula: 

are both uncentered standard deviations assuming zero mean of the population, hence they 
are marked with superscript ‘(0)’. The higher the
$v_r$, the more similar the expression profile of a gene is to a module’s mean expression 
profile. For ranking genes, there are three options provided based on the (uncentered) correlation 
score calculated. First one is ‘correlation’ which uses $v_r$ directly as final score. The 
second option ‘Absolution correlation’ takes $v_r$. It ranks both correlated and anti-correlated 
genes based solely on their similarities. Instead, the third option ‘Anti-correlation’ take 
$v_r$ as score to favor only the anti-correlated genes.

\subsection{Gene variability}

The variability of a gene's expression value $x$ for conditions $i=1,…,n$ is calculated as the 
uncentered standard deviation: 





\section{Appendix B: Magic processing}




%%%%%%%%%%%%%%%%%%%%%%%%%%%%%%%%%%%%%%%%%%%%%%%%%%
% Keep the following \cleardoublepage at the end of this file, 
% otherwise \includeonly includes empty pages.
\cleardoublepage

% vim: tw=70 nocindent expandtab foldmethod=marker foldmarker={{{}{,}{}}}
