\chapter{Abstract}                                 \label{ch:abstract}

%\ldots

\instructionsabstract

%% We first thoroughly studied the issues involved in creating
%% cross-platform compendium.
%% % data collection
%% First, special attention has been paid on studying data representation
%% heterogeneity, which has been widely recognised yet lack of further
%% investigation.
%% %
%% Several aspects that complicate the data collection task, particularly for
%% those generated on dual-channel microarrays, are identified, and derived
%% specific strategies to tackle each of them.
%% %
%% Based on our study, a semi-automatic system is developed that enables
%% efficient and high-quality expression data retrieval over a broad range of
%% platforms, and consequently, makes it feasible to create large-scale
%% compendium.
%% % annotation
%% Next, a manual annotation step is taken to resolve the issue with the
%% meta-data incompleteness and inconsistency, providing expert-curated and
%% consistent specifications of experimental factors and sample attributes
%% that are crucial for data exploration and result interpretation.
%% % homogenization (still missing)


%% -- Advantage 
%
%% The methodology has maken it feasible to construct sizable compendia for
%% species that lack of dominant microarray platform, and helped creating a
%% compendium that is far more comprehensive than the existing
%% single-platform one.

%% % from introduction v1
%% %
%% It is composed of three major steps: data collection that collects raw
%% expression data of different experiment and platform origin overcoming the
%% wide-spread representation heterogeneity issue; annotation that aims to
%% provide the consistent and complete specification of experimental factors
%% and sample attributes crucial for data exploration and result
%% interpretation; and at last, data homogenization in which raw data are
%% preprocessed using our in-house pipeline to improve consistency and
%% log-ratios are calculated to render the data comparable across different
%% platforms.


% Transcriptome - microarray - compendium

With the unprecedented ability to systematically probe gene expression
at genome scale, microarray has become an indispensable technology
adopted by every laboratory across the world, generating a colossal
amount of data for a variety of species.
%
Although comprehensive at gene dimension, any microarray based study
alone provides a limited scope to study specific aspect(s) of an
complex organism through analyzing its transcriptome.
%
Alternatively, the aggregate of existing expression data provides an
opportunity to investigate gene expression of a particular species at
a global level, and alternatively, view a specific study from the
perspective of existing knowledge.
%
The goal of this research focuses on developing novel methodology and
systems to explore this opportunity.
%
%% construct such an exhaustive data set called compendium from publicly
%% available expression data, and creating appropriate tools to explore it.



We first developed a methodology to create an organism-specific
cross-platform compendium from publicly available gene expression data.
%
Special attention has been paid on two aspects, resolving data
representation heterogeneity to facilitate automated data retrieval
across data generated on different microarrays, and improving data
consistency and compatibility through systematic renormalization of data
and subsequently, log-ratio calculation.
%
%% Compared with this novel methodology, the existing approach, which
%% constructs compendium from data generated on single-platform, is
%% inferior with its limited applicability and the restrained scope of the
%% resulting compendium.
%
Compared with the existing approach that constructs compendium from
the data generated on single-platform, this novel methodology is
superior with its broader applicability and the extensive scope of the
resulting compendium.


Utilizing this novel methodology, we constructed three such
comprehensive expression compendia for the bacterial model organisms
(\textit{Escherichia coli}, \textit{Bacillus subtilis}, and
\textit{Salmonella enterica} serovar Typhimurium).
%
Moreover, efforts have been taken to create a web access portal
providing public access to these three compendia.
%
Especially, a suite of intuitive tools suitable for exploring,
analyzing, and visualizing expression data of such cross-platform
compendium are incorporated into this web portal to promote the utility
of the compendium.
%
Employing these tools to explore \textit{E. coli} compendium, we
successfully identified novel targets for transcription factor FUR
(Ferric Uptake Regulation) based on their expression similarity to that
of the known FUR targets across a range of conditions of different
experiments and platforms origins.
%
%% The utility of both the compendia and the web portal is demonstrated in a
%% case study, in which COLOMBOS analysis tools are employed to identify novel
%% targets for \textit{E. coli} transcription factor FUR (Ferric Uptake
%% Regulation) based on their expression similarity to that of the known FUR
%% targets in compendium


%% One of the most important applications of compendium is to discover
%% condition dependent co-expression modules, which is utilized to study
%% the regulatory program of coexpressed genes and to infer regulatory networks.
%
%%   Carolina: better a more biological description (below), instead of
%%             too technical (above) ...
%
One of the most important applications of compendium to study a
organism's response to the genetic and environmental changes by
identifying condition dependent functional modules, and studying the
underlying regulatory mechanism responsible for the observed expression
variations.
%
Many methods exist for this purpose. Each makes different assumptions to
handle the under deterministic nature of this complex problem, and
consequently generates complementary results.
%
Here, we demonstrated such complementarity between two methods,
DISTILLER and COLOMBOS, in a case study, in which co-expression modules
containing gene \textit{sodA} are extracted from \textit{E. coli}
compendium using each method, and compared against each other.
%
Through this example, we stress the importance of choosing the right
method based on the research purpose.


%% Initially, we focused our efforts on less complex unicellular
%% prokaryotic microbes.  However, such a cross-platform approach has the
%% great potential to create comprehensive compendia that benefit
%% researches on non-momdel eukaryotic organisms, especially those of great
%% economic interests and accumulated a substantial amount of public
%% expression data across different platforms.
%
% Therefore, 
At last, we extended the methodology to handle the extended complexity
of monocot \textit{Zea mays}, specifically addressing two issues,
platform probe annotation inconsistency and precise biological sample
annotation that manifests the abundant genetic repository (breeding
lines) of maize species, and its complex life style (development stage)
and structure (tissue).
%
We also upgraded web access portal accordingly with new functions to
facilitate the utility of rich annotations available.




%%%%%%%%%%%%%%%%%%%%%%%%%%%%%%%%%%%%%%%%%%%%%%%%%%
% Keep the following \cleardoublepage at the end of this file, 
% otherwise \includeonly includes empty pages.
\cleardoublepage

% vim: tw=70 nocindent expandtab foldmethod=marker foldmarker={{{}{,}{}}}
