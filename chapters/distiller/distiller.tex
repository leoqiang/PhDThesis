\chapter{Directed module detection in a large-scale expression compendium}\label{ch:distiller}
\chaptermark{Gene coexpression module discovery}


\instructionsintroduction

% specially indented text in bold 
\newcommand{\distsubparagraph}[1]{\hspace{-2mm}\textsf{\textbf{#1}}}

%\todo{[distiller]Check all the links and update them}

%\textbf{Abstract}
%Public online microarray databases contain tremendous amounts of expression 
%data. Mining these data sources can provide a wealth of information on the 
%underlying transcriptional networks. In this chapter, we will illustrate how 
%the web services COLOMBOS and DISTILLER can be used to identify condition 
%dependent coexpression modules by exploring compendia of public expression 
%data. COLOMBOS is designed for user-specified query-driven analysis, whereas 
%DISTILLER generates a global regulatory network overview. The user is guided 
%through both web services by means of a case study in which condition 
%dependent coexpression modules comprising a gene of interest (i.e. `directed') 
%are identified.

\section{Introduction}

Omics based approaches are increasingly being used to uncover underlying 
mechanisms of bacterial behaviour \cite{Fierro2008}. The obligate deposit of 
high throughput experiments in public databases upon publication has 
tremendously increased the amount of publicly available experiments. Mining 
these data sources helps in gaining a global condition dependent view on 
bacterial gene regulation \cite{Lemmens2009, Fadda2009}, in expanding the 
current knowledge on transcriptional interactions with novel reliable 
predictions \cite{Faith2007, Ernst2008}, and in comparing transcriptional 
networks across species \cite{Fierro2008, Babu2009}. It also offers molecular 
biologists the possibility to see their own dedicated analysis in light of what 
is already available. Inferring transcriptional networks from these public data 
usually requires complex normalization procedures \cite{Faith2008} and 
computationally intensive algorithms. To enhance the usability of these tools, 
they are often wrapped in a web service.

In this chapter, we will illustrate how compendia of public expression 
data can be used to identify condition dependent coexpression modules, in which 
a particular gene of interest is involved (`directed' module detection), by 
means of two web services: COLOMBOS \cite{Engelen2011} and DISTILLER 
\cite{Lemmens2009} (the latter in combination with the visualization tool 
ViTraM \cite{Sun2009}). Figure \ref{fig:workflow-distiller-colombos}
illustrates the difference between both approaches. COLOMBOS only relies on 
expression data to retrieve condition dependent modules, implying that there is 
a functional relationship between the module genes, but not necessarily that 
they are regulated by the same (set of) transcription factor(s). DISTILLER, in 
contrast, incorporates additional motif data and the constraint that module 
genes should share the same regulatory program, implying that the module's 
condition dependent coexpression can be directly linked to transcriptional 
coregulation. 

To clearly demonstrate the functionalities of these web services, 
the gene \textit{sodA} is used as a case study. This gene encodes for a protein 
with superoxide dismutase activity, which reduces harmful free radicals of 
oxygen formed during normal metabolic cell processes \cite{Fawcett1995, 
Tardat1993, Jair1995}. 
It is known to be regulated by several regulators, such as, 
Fur, SoxS, and MarA. As such its expression is coupled to different biological 
conditions: multiple antibiotic resistances (MarA), superoxide resistance 
(SoxS) and the intracellular iron pool (Fur). By applying different analysis 
approaches (COLOMBOS and DISTILLER), we will show how to identify \textit{sodA} 
containing modules, i.e. genes that behave similarly to \textit{sodA} in a 
condition dependent way from large-scale expression compendia.

\begin{figure}[tb]
	\centering
  	\includegraphics[width=1\textwidth]{Fig1-Workflow-figure.png}
	\caption[Workflow of directed module detection in expression 
	compendium]{
	\textbf{Workflow of different methods for the directed module detection in 
	expression compendium.} 
	Coexpression modules are detected by COLOMBOS starting from a user provided 
	seed gene (e.g., \textit{sodA}). 
	DISTILLER searches for coexpression modules in a global fashion 
	(information on seed genes is not a prerequisite) and is constrained by 
	regulator-to-gene assignments. 
	Coexpression patterns retrieved by COLOMBOS point toward a functional 
	relationship among the module genes but do not necessarily imply 
	coregulation. To gain insights on the regulatory program of those genes, de 
	novo motif detection tools could be applied to analyze their promoter 
	regions. Inherent to the regulator-to-gene constraints implemented in 
	DISTILLER, there is a direct link between coexpression and coregulation for 
	the genes in the modules that it retrieves.}
	\label{fig:workflow-distiller-colombos}
\end{figure}

\section{Material}


\subsection{Cross platform expression compendium}\label{sec:dist-comp} 

An expression compendium is essentially an organism-specific matrix of
expression values derived from publicly available microarray experiments which
are homogenized to make them comparable. The rows of the matrix correspond to
the known genes of the organism in question.
%
Each column is a \textit{contrast} defined as a comparison between two different
biological samples, one acting as a test and the other as a reference.
%
The expression values are calculated as expression log-ratios representing gene
expression changes induced by the difference between samples.
%
Relative expression calculated intra-experiment/platform (i.e. between two
conditions measured in the same microarray experiment using one platform)
negates much of the platform and experiment specific variations that makes it
impossible to reliably compare the absolute quantities reported in different
experiments \cite{Shi2006}.
%
The extensive annotations are manually curated for each contrast, specifying
which aspect(s) (experimental conditions) has/have been changed between the test
and reference samples.
%
The methodology utilized to create such a compendium is explained in details in
Chapter \ref{ch:command}.


%% The cross platform expression compendium consists of a large amount of publicly 
%% available microarrays derived from different platforms (e.g. cDNA microarrays, 
%% Affymetrix). 
%% Data were normalized to remove cross platform differences and 
%% experimental conditions were annotated in a formal manner (see Section 
%% \ref{sec:colombos-comp}).
%% The expression data are presented as expression log ratios of measurements in a 
%% test condition versus a reference condition.
%% A \textit{condition contrast} is the set of all 
%% genes' expression log ratios, each of which represents the log ratio of a 
%% gene's expression values obtained in one particular experiment for which a test 
%% is compared to a reference conditions. 
%% The condition annotations of each contrast thus indicate which aspect(s) 
%% (experimental conditions) has/have been changed between the test and reference 
%% samples.  
%% Please refer Section \ref{sec:colombos-comp} for more details about 
%% the compendium and its condition annotations.



The \textit{Escherichia coli} compendium used in here contains 1429 condition
contrasts obtained from 1747 microarrays of 84 experiments across 35 platforms,
covering the expression profiles of 4295 genes under various conditions
annotated by 242 different condition properties grouped under 56 condition
ontology terms.
%
Furthermore, several sources of gene information from main curated databases are
integrated into the compendium as well, such as, the pathway and operon
information from EcoCyc \cite{Keseler2009}, the Gene Ontology annotation from
UniProt GOA \cite{Camon2004}, and the regulon information from RegulonDB
\cite{Gama-Castro2008}. (see Table \ref{tab:colTB-overview})

%The condition annotation is specified using a set of formal condition 
%properties (representing for instance mutations, compounds in the growth 
%medium, treatments, and general growth conditions) and associated absolute 
%values for these properties (for instance, concentration of a certain compound 
%in the growth medium). As such, we can specify the condition annotation of 
%each 
%condition contrast rigidly as a vector representing the differences between 
%these property values in test and reference condition for each of the relevant 
%condition properties (for instance, the concentration difference between test 
%and reference condition). This enables the mathematical comparison of contrast 
%annotations. For instance, a contrast condition annotation specified as 
%`FeSO$_4$: 50$\upmu$M' describes that for this condition contrast, the 
%concentration of FeSO$_4$ has changed by a quantity of `50$\upmu$M', in the 
%medium of the test versus the reference sample. Due to the large number of 
%different properties and their distinct nature, it may not be easy for a user 
%to find those condition properties he or she is interested in. To facilitate 
%this, we have provided a condition ontology similar to Gene Ontology 
%\cite{Ashburner2000}. This ontology maps the condition properties used to 
%annotate the condition contrasts to the biological processes or 
%functionalities 
%they most likely affect. Each property can be 
%mapped to one or more ontology terms and thus ensures an intuitive, higher 
%level view of the condition contrasts. As an example, the condition ontology 
%term `amino acid metabolic process' includes condition properties that are 
%linked to amino acid metabolism, such as the addition of arginine to the 
%medium, \textit{trpR} mutation (tryptophan transcriptional repressor), etc.  
%The condition ontology superimposed on the condition annotation thus 
%represents 
%a broader, more intuitive view of the altered stimuli that are responsible for 
%the observed expression changes (i.e. log ratios) of the condition contrasts. 


\subsection{Coexpression modules}

Genes that have the same regulatory program (here defined as having the same 
transcription factor binding motifs in their promoter regions) behave 
similarly: their expression responds in a similar manner to certain alterations 
in the organism's intra- or extracellular environment. This is called 
\textit{coexpression}. The evidence of coexpression points to a functional 
relationship between coexpressed genes, and might sometimes be an indication of 
shared regulatory programs. 
%
Such a functional relatedness is only enforced when coexpression occurs across
multiple different conditions. Since each value in our data set does not
represent an absolute measure of expression, but rather a relative one
associated to a contrast (test versus reference), coexpression in this case
translates to genes showing log ratios that are significantly different from
zero for at least one condition contrast, and for each of the involved contrasts
show coherent changes in the same direction (either up or down).
%
The methods, COLOMBOS and DISTILLER, demonstrated in this chapter employ
different computations to score this coexpression, but essentially look for the
same phenomenon.

Genes are only coexpressed under certain condition contrasts. This combined set
of genes and condition contrasts, where the coexpression pattern appears, is
what we refer to as a \textit{module}. Extra constraints can be used to tune the
module concept for specific purposes. For example, extra requirements could be
added that genes in the module should have (known) motif(s) in common.


\subsection{COLOMBOS}

COLOMBOS, an acronym for \textit{COLections Of Micrcroarrays for Bacterial
  OrganismS}, is a web service \cite{COLOMBOS, Meysman2014}
(Figure \ref{fig:colombos}) for interactively exploring, querying, analyzing and
visualizing data from cross platform expression compendia through an intuitive
interface.
%
Currently it provides the possibility to query expression compendia based on the
annotation of their contrasts and/or genes.
%
This service can be used, for instance, to search for genes being coexpressed
with one or more genes of interest under a pre-specified set of condition
contrasts, or to search for the condition contrasts in an expression compendium
under which a pre-specified gene set is coexpressed. As such, it can identify
coexpression modules based on user input.
%
In COLOMBOS, modules can be visualized as interactive heatmaps, showing the
annotation of genes and condition contrasts as obtained from public databases
and corresponding literature. More details of COLOMBOS are discussed in Chapter
\ref{ch:colombos}.

\begin{figure}[tb]
	\centering
  	\includegraphics[width=1\textwidth]{Fig2-COLOMBOS.png}
	\caption[COLOMBOS web service interface]{\textbf{COLOMBOS web 
	service interface.} 
	The interface is composed of three parts: the website navigation header at 
	the top, the workspace (showing a module tree) at the left-hand side, and 
	the operating space (here showing the `Data analysis overview' panel) 
	filling the center.}
	\label{fig:colombos}
\end{figure}

Currently COLOMBOS (v2.0) runs on three browsers: Firefox, Google Chrome,
Safari, and Opera. Be sure that you are using the latest version of the browser
to have the best compatibilities. Please check the website for updated
information.
% Please refer to \textbf{Note 1} for version and supported browsers of the 
%website.

\subsubsection{COLOMBOS expression data format}\label{sec:dist-format-col}
To use COLOMBOS requires no input file, as the expression compendium is an
integrate part of It.  
%
It does provide a tab delimited ASCII file format for expression data export,
containing information on the user created module consisting of a particular set
of genes and the corresponding set of contrasts under which these genes are
coexpressed.
%
There are two sections in this file. The first one describes the condition
information of the module. It specifies for each contrast the name and
description of its test and reference sample, the corresponding experiment
identifier, the online database from which the experiment was retrieved, the
microarray platform used to measure the sample hybridizations, and finally, the
annotated condition changes between the test and the reference sample.
%
The second section contains three parts: the corresponding contrast identifiers
in the first row; the gene information (gene locus tag, gene name, and internal
gene identifier) in the first three columns; and the expression log-ratios of
the compendium.
%
Starting from the $4^{th}$ column, each column in the file corresponds to one
condition contrast. Each of the rows represents one gene and its expression
values for different condition contrasts.


\subsection{DISTILLER}\label{sec:dist-distiller}
DISTILLER (\textit{Data Integration System To Identify Links in Expression
  Regulation}) is a data integration framework that searches for condition
dependent transcriptional modules by combining expression data with information
on the interaction between a regulator and its corresponding target genes, for
instance based on motif screening. A module is here defined as a set of genes
coexpressed under a sufficiently large number of conditions and sharing the
motif instances of the same regulator(s).  The expression data used by DISTILLER
could be either absolute expression values or log-ratios.

DISTILLER searches the modules in a global fashion over the entire dataset
through three steps: identification candidate modules, module filtering, and
module extension.
%
Built on advanced itemset mining approaches, it first exhaustively enumerates
all possible valid \textit{closed} modules (see next Section) as candidates. The
candidates are partially redundant as they might share the same genes or
condition contrasts.
%
Therefore, in the filtering step, modules are prioritized by calculating an
interest score for each of them.  The score takes into account the probability
that a module of the same size can be found by chance in the input dataset,
whereas, at the same time, penalizes the overlaps between modules.
%
The resulting top ranking modules are statistically significant yet distinctive.
These initial modules obtained under stringent thresholds are called
`\textit{seeds}', in which the identified interactions between a transcription
factor and its target genes are highly reliable (high precision).  However, some
true interactions might be missed (lower sensitivity).
%
Hence, in the last module extension step, the modules retained after filtering
are further extended with additional genes by applying relaxed thresholds on
minimal coexpression level and motif score.
%
Note that it is not required to execute all three steps.  The result of each
step can be analyzed and visualized independently.

The DISTILLER web service \cite{DISTILLER} (Figure \ref{fig:distiller}) is 
designed to leverage the difficulty of applying the algorithm by providing an 
easy-of-use and consistent interface that links all three steps together. 
%
The interface is separated into two sections vertically. The information panel
at the top provides a brief explanation of the site's functionality. The project
panel at the bottom is where the user interacts with the site.
%
It is further divided horizontally into three sections.  At left, the `Project
Information' panel provides information of the current project and the links to
access the input, output, and log files.
%
The `Operation' panel at the center is where a user runs DISTILLER algorithm.
It has three tabs, each of which provides access to one step of the algorithm,
namely, module detection, module filtering, and module extension.
%
The explanation for the parameter settings and the input data file format (if
required) of the corresponding step selected in the `Operation' panel are
provided in the `Online help' panel at right for your reference.  
%
The website works properly on Firefox and Google Chrome.  Efforts are undertaken
to make it compatible with other browsers as well.  Please check the website for
updated information. % For compatibility issues,
%please refer to \textbf{Note 1}.

\begin{figure}[tb]
	\centering
  	\includegraphics[width=1\textwidth]{Fig3-DISTILLER.png}
	\caption[DISTILLER web service interface]{\textbf{DISTILLER web service 
	interface (at module detection step).}}
	\label{fig:distiller}
\end{figure}

For this case study, the data consist of the same compendium as the one 
accessed through the COLOMBOS website. Hence, the columns of the expression 
matrix, across which DISTILLER defines coexpression, correspond to the 
condition contrasts as defined in the compendium (see Section 
\ref{sec:dist-comp}). 

\paragraph{Closed module}
A module is considered \textit{closed} if it cannot be further extended by any
other gene without reducing the number of motifs shared by all of its genes. To
search such a candidate, the algorithm \cite{Lemmens2009} starts with the
smallest module that contains only one gene, and gradually extends them by
merging with other modules. A candidate is found when any extension violates the
conditions implied by `closed'.  Only closed modules that contain the minimal
number of motifs and condition contrasts are considered valid.
% (meet the required minimal Supports on the motifs and condition contrasts)



\subsubsection{DISTILLER input file format}\label{sec:distiller-infile-format}
DISTILLER requires two input files: an expression data file and a motif data 
file. In the case study described here, the expression data file contains the 
expression log-ratios for each gene under each contrast. 
%
Two possible input formats are allowed for the expression data file.  The first
one is a matrix based flat file format with header, in which, the first row
contains the identifiers of each contrast, each subsequent row corresponds to a
gene, each column represents a contrast.  Gene name is used to specify a gene in
the first column of each row.
%
The second format is that of the COLOMBOS expression data export file (see
Section \ref{sec:dist-format-col}).  
%
The system further supports the use of two most common compression formats to
handle the large data set, namely, Zip\footnote{In case of zip file, the system
  allows only one file in the compressed file that is the input expression data
  file. An error is generated when this condition is violated.} and Gzip.


The motif data file contains the results of a motif screening analysis 
\cite{Hertzberg2005}. 
%
Each motif screening score indicates the probability that a motif instance is
present in the promoter region of a gene, with 1 as the maximum score and 0 the
minimum.
%
The data is specified in a text file containing a data matrix with header, in
which the header are motif names, the row corresponds to gene, and the column
represents the motifs screened.  Similarly to the expression data, except for
the header, each row starts with gene name specifying the corresponding gene.

Additionally, users can provide an operon information file that will be used
during the module extension step.  It contains two items in each row, the gene
that belongs to an operon, and an identifier of that operon.

In DISTILLER, the different data files are coupled in the gene direction, so the
user should make sure that the same gene identifiers are used in each file,
although the order of the genes need not to be the same.


\subsubsection{DISTILLER output file format}\label{sec:distiller-outfile-format}
All three steps of DISTILLER output the results in the same format: a `.m' file.
%
It contains the information of the identified modules, with each module
described in a section, separated by an empty line.  
%
Each section contains four data fields.  Given a section specifying the
information for a module $M$, the field `Significances' holds the $p$-values
that were assigned to $M$ by the itemset mining algorithm \cite{Zaki} utilized
by DISTILLER.  The field `items' contains the information of the genes that
belong to $M$, each of which is represented by its row number\footnote{When
  counting the row number of a given gene, there are certain caveats. First, the
  header should not be counted.  Second, the count should start from 0. Hence
  the first row is of row number 0 instead of 1.} in the expression data input
file. The field `boxtidset1' specifies the contrasts of $M$, in which the genes
in $M$ are coexpressed, each of which is represented by its column
number\footnote{To count the column number referred in the `.m' file, the first
  column containing gene information should not be counted, and it starts at 0.}
% \footref{fn:dist-count} 
in the expression data input file. Finally, the field `tidset1' contains the
motifs of $M$ shared by its genes, specified by their column numbers in the
motif data input file.  
%
Each module is represented by a unique number indicated between the `\{\}'
brackets after the name of each corresponding data field.

Besides the main output file, two supplementary files required to use the
visualization software ViTraM are also provided. They are bundled in a
compressed file that can be downloaded from the website. The file whose name
starts with `expdata\_' contains the preprocessed input expression data, and the
other starting with `binary\_' contains the preprocessed input motif data. The
file formats of these two files are the same as the corresponding input files
described in previous Section.


\paragraph{Output notification email}\label{sec:distiller-email}
Since each step of DISTILLER may take hours or even days to finish, the result
is sent to the user by email. This notification email is of a standard format,
with subject `DISTILLER process result notification email'. On the first line,
the information about the finished process and the corresponding user project is
presented, followed by the link to the result file, then the link to the
supplementary file bundle required for visualization.


\subsection{ViTraM}\label{sec:dist-vitram}
To analyze DISTILLER output, we use ViTraM (\textit{VIsualization of
  TRAnscriptional Modules}) to visualize overlapping transcriptional
coexpression modules together with the motif information in an interactive
way. Here, we will only briefly discuss this tool. For more details we refer to
\cite{Sun2009}.

ViTraM 2.0 is used for this case study (download at \cite{ViTraM}).  Written in
JAVA, it can run under any platforms with JAVA support.  On Windows ViTraM comes
with two options: vitram\_Windows\_512M.bat or vitram\_Windows\_2G.bat. The
former has a minimum memory requirement of 1Gb RAM.  And we advise to use a
computer with at least 3Gb RAM to run vitram\_Windows\_2G.bat. 
% Please refer to \textbf{Note 1}
% for compatibility issues and memory requirements of the tool.


\subsubsection{ViTraM input file format}
ViTraM requires an XML file and an expression file as input. The XML file
contains the information of the transcriptional modules that will be visualized:
the genes and conditions composing the modules and additional information such
as motifs that are assigned to modules by DISTILLER. Due to the complexity of
the XML format, an extra tool called XMLCreator is available at the ViTraM
website to automatically generate the XML file from the DISTILLER output file
% and the supplementary expression data file of DISTILLER 
(see Section \ref{sec:distiller-outfile-format}).  Conveniently, the tool also
generates a smaller expression data file containing only the relevant expression
values from those genes and condition contrasts existing in the result modules
obtained by DISTILLER.  Moreover, the tool can also incorporate extra
information into XML file, for example, the motif data contained in the binary
supplementary file mentioned in Section \ref{sec:distiller-infile-format}.
%
For the purpose of this study, we will not discuss this XML file format in
detail.  Interested users can find more information in \cite{Sun2009}.

%\subsection{Input file formats}\label{sec:dist-format-input}
%
%Different file formats are used by COLOMBOS, DISTILLER and ViTraM. Depending 
%on 
%their functions, the input files are explained in this section and the output 
%files in the next one. Sample files that contain the data used for our case 
%study are also provided (see Section \ref{sec:dist-sample}). 
%
%
%\subsection{Output file formats}
%
%Below, the different output file formats are explained in each subsection.






\subsection{Sample files}\label{sec:dist-sample}

The following four sample files are provided as example input files for this 
case study. The file expdata\_COLOMBOS\_module\_information.txt is an example 
of the exported module data file of COLOMBOS, which, as one of the input 
formats of the expression data for DISTILLER, contains both the gene and 
condition contrast information, and the log ratio expression data. The file 
expdata\_DISTILLER\_Expression.txt is an example of the other expression data
format accepted by DISTILLER. The file binary\_DISTILLER\_Motif.txt is the motif
data containing motif scores for each gene. And finally, the file 
operon\_DISTILLER\_OperonGenes.txt is the example operon information file.

Furthermore, three example output files, corresponding to the output generated
after each step of DISTILLER, are provided. DISTILLER\_Output.m is the output of
the seed module identification step, DISTILLER\_FilteringOutput.m is the output
of the module filtering step, and DISTILLER\_outputExtended.m is the output of
the module extension step. Files expdata\_DISTILLER\_supple.txt and
binary\_DISTILLER\_Motif\_supple.txt are provided as the supplementary files 
used for visualizing modules contained in those sample output files.

Finally, ViTraM\_Modules.XML is provided as the XML file describing the modules,
and ViTraM\_Expression.txt is provided as the expression data file used by
ViTraM.

One can download all the example files from here \cite{DISTILLER-sample} or at 
the DISTILLER website \cite{DISTILLER} (follow the sample file link of the 
second references).




\section{Methods}

We will describe the steps required to use two web services, COLOMBOS and
DISTILLER, to identify coexpression modules around a (set of) gene(s) of
interest (query genes). This will be illustrated by applying them to the query
gene \textit{sodA} as a case study.  
%
The analysis flow based on COLOMBOS shows how it can be used to search for
genes that are coexpressed with a (set of) query gene(s) under the human
guidance.  Alternatively, the DISTILLER approach, combining expression and
motif data, detects coexpressed and coregulated gene sets across the whole
data set in a single run in an unsupervised way.
%% It searches for gene sets that are coexpressed and contain the same
%% motif(s) in their promoter region by combining expression and motif data.


\subsection{Identifying coexpression modules using COLOMBOS}\label{sec:dist-module-colombos}

To search for coexpressed genes using COLOMBOS, we will first create an initial
expression module by specifying a set of genes of interest (query genes), and
use COLOMBOS to extract the most relevant contrasts, where the chosen genes are
differentially expressed.  Next, the module is extended with genes that are
coexpressed with the initial gene set under the selected of contrasts. Here, we
explain this workflow starting with a single query gene, \textit{sodA}.

\subsubsection{Create the initial module}

\begin{small} % Step 1

\paragraph{Step 1} 
Go to COLOMBOS website \cite{COLOMBOS}, and create the initial module based on
the known gene set of interest and the biological conditions that are known to
affect these genes when changed.

\subparagraph{Step 1.1}	To start the analysis, click on `data 
analysis' in the title bar (Figure \ref{fig:colombos}) to bring up the data
operation interface in the center part of the website.  It is separated
horizontally into two parts. At the left hand side, there is a `workspace'
information panel that lists all the modules created in a tree structure. This
panel is always visible when residing on the data analysis page.
%% (Note that you can always move away from the data analysis 
%% page to any of the other COLOMBOS pages, such as the help sections, at any 
%% time; the state of the data analysis page will be preserved for whenever you 
%% switch back.)
Since no module has been created, it is currently empty.
%
The other part of the page, referred as the operating space in this text, is
where the visualization and analysis of the expression data takes place and is
currently occupied by the `Data analysis overview' panel (hereafter referred to
as `overview panel'). This panel serves as a intuitive guide showing the
available analysis options depending on the current workspace state along with
some brief explanations.\vspace{1.5mm}
%% The `Main' button at the top right of the workspace will always bring up the
%% overview panel again in the operating space.
%
Since we just started a new session, there is only one button `select organism'
available. Click it, and select the organism `\textit{Escherichia coli}', which
this case study focuses on. Notice now this species appears in the workspace
panel as the root of the tree.


\subparagraph{Step 1.2}	After the species is picked, the content of the 
overview panel is changed accordingly. The button `select organism' is replaced
by two other buttons `organism info' and `add module'.
%% The workspace panel now also shows that \textit{E. coli} data is loaded. 
%% Clicking on the \textit{E. coli} entry in the workspace panel has the same 
%% function as clicking the `organism info' button in the operating space: it will 
%% bring up detailed information about the organism, its compendium, and 
%% associated additional data sources. 
%
To create our initial module for \textit{sodA}, we click the `add module' button
to bring up the `Add module overview' panel.
%% We will proceed to create our initial module of the gene \textit{sodA}. The
%% `Add module overview' panel appears in the operating space after clicking the
%% `add module' button.
%
In this panel, there are three buttons `select gene only', `select contrasts
only', and `select gene and contrast'.  Each corresponds to a different method
for creating a module.
%
Here, we utilize the first option to manually specify the query gene
\textit{sodA} and let the system automatically identify the most relevant
contrasts for it based on expression data.
%% For this case study, we will only manually specify the gene \textit{sodA}
%% without specifying any relevant conditions ourselves.  Instead, we will have
%% COLOMBOS automatically identify the most relevant condition contrasts for
%% sodA based on its expression values.  This function is provided by the
%% leftmost button `select gene only'.
%
Clicking `select gene only' brings up the gene selection panel at center.  Four
available options are available: `By gene name/locus tag', `By transcriptional
regulation', `By pathway', and `By transcription unit'.
%
The first option allows us specify gene manually.
%% We want to select a specific gene (\textit{sodA}), so the first option is our
%% choice.
Click the yellow box surrounding the text `By gene name/locus tag', fill in
`\textit{sodA}' in the text box appeared (one gene per row), and then click the
`Done' button to proceed. %save your input.
%% If one or more of the query genes is not recognized or included in the
%% compendium, an error message will appear listing the unresolved entries.

\subparagraph{Step 1.3}	
%% As mentioned in previous step, we chose the option `select genes only' to
%% create the module.
%
\begin{figure}[b]
	\centering
  	\includegraphics[width=1\textwidth]{Fig4-Condition-contrasts-ranking-panel.png}
	\caption[COLOMBOS contrast ranking interface]{\textbf{The `ranked contrast 
	selection' interface to choose contrasts based on the specified module 
	genes (\textit{sodA} in this particular case).}}
	\label{fig:colombos-ranking}
\end{figure}
%
After specifying the genes, a `ranked contrast selection' panel
(Figure \ref{fig:colombos-ranking}) appears in the operating space, which allows
the user to select contrasts ranked by a score that prioritizing those show
the highest magnitude of change and most coherent coexpression for the
selected set of genes (see Appendix \ref{apd:contrast-score}). The panel is
separated into two parts horizontally.  At the left, the contrasts are ranked
according to the score from highest till the lowest. A cutoff value can be
specified in the box above the list to select only those with higher
scores. The figures to the right show, from left to right, a density plot
showing the distribution of the scores across the whole compendium and a plot
showing the number of contrasts that will be added if a given cutoff is
selected.  There is no absolute guideline to choose the cutoff. Several
try-outs may be needed to identify the optimal one.
\\
For our initial module, we chose cutoff value 3. Specify it, and click `Ok'
button at the side. The contrast list is then filtered, and all that left in
the list are automatically selected to be added to the module. Click `Done'
button at the bottom, and fill in `sodA\_c3' as the module name to create our
initial module.  The newly created module appears in the module tree in the
workspace as a node directly under the root (presenting the selected organism
\textit{Escherichia coli}).

%% The module `sodA\_c3' is now created and will appear in the module tree in the 
%% workspace at the left side as a node directly under the root node (presenting 
%% the selected organism \textit{Escherichia coli}).

\end{small} % Step 1



\subsubsection{Inspect the existing modules}

\begin{small} % Step 2
\paragraph{Step 2} One can review the information of a module through various 
options.
%% After a module is created, one can review its information through various 
%% available options.

\subparagraph{Step 2.1}	Click the module you want to check in the workspace 
to show its overview tab (Figure  \ref{fig:colombos-init-module}) in the center.
The tab shows on top a brief summary of the module, including, name,
description, number of genes and contrasts it contains, as well a list of
significantly enriched Gene Ontology (GO) terms of class biological process
($p$-values $< 0.1$, details see Appendix \ref{apd:enrichment}) for the module
genes.
%% It provides a brief summery of the current module, and serves as a helping
%% guide to analyze and visualize the module's composition.  For each module,
%% a summary section shows its name followed by its description at the
%% top. Below the number of included genes and contrasts are indicated, as
%% well a list of Gene Ontology (GO) enrichment scores (see Appendix
%% \ref{apd:enrichment}) for the genes in the module. Only the GO terms of the
%% class biological process are calculated for their enrichment scores, and
%% only those with $p$-values $< 0.1$ are shown.
%
\begin{figure}[tb]
	\centering
  	\includegraphics[width=1\textwidth]{Fig5-Module-overview.png}
	\caption[`sodA\_c3' module overview]{\textbf{The overview tab of 
	the module `sodA\_c3'.}}
	\label{fig:colombos-init-module}
\end{figure}
%
At the bottom, there are four buttons to \textit{visualize}, \textit{edit},
\textit{split}, or \textit{delete} the module. The function of
\textit{visualization} and \textit{delete} is straightforward. The
\textit{edit} function allows user to modify either the genes or the contrasts
of the module. Instead, the \textit{split} function breaks the current module
into several new ones by separating either its genes or its contrasts into
distinctive groups and creating a module for each selected group.


\subparagraph{Step 2.2}	Each module can be visually presented as a heatmap. 
In it, each square represents one gene's expression log ratio for a certain
condition contrast. A red color indicates over-expression while a green one
shows under-expression. The intensity of the color corresponds to the
magnitude of the expression change. The brighter the color, the higher the
absolute ratio value.
%
To view the heatmap, click `Visualization' button in overview tab to switch to
heatmap tab.  The heatmap is shown at the center with an information panel at
the right. Hovering over the heatmap, the information of the gene and the
contrast under the cursor will be shown in the information panel. Various
options exist to group the contrasts in the heatmap according to their
experiments, condition properties, and the associated condition ontology
terms. They can be selected from the dropdown box located at the lower left
corner of the tab. The heatmap of our module sorted on condition ontology is
shown in Figure \ref{fig:colombos-heatmap-m1}.
%
\begin{figure}[tb]
	\centering
  	\includegraphics[width=1\textwidth]{Fig6-Heatmap-module-1-white.png}
	\caption[Heatmap of module `sodA\_c3']{\textbf{The visualization tab of 
	module `sodA\_c3' showing the heatmap.}
	In this heatmap, the contrasts are grouped by their ontology annotations. 
	The groups belonging to the ontology term `response to oxygen 
	level' are marked out.}
	\label{fig:colombos-heatmap-m1}
\end{figure}

\end{small} % Step 2

Checking module `sodA\_c3', it shows that the top 53 contrasts, where
\textit{sodA} is most differentially expressed, have been selected. Among
those, 34 contrasts belong to the condition ontology term `response to oxygen
levels'.  It includes condition properties that are linked to cellular
processes dependent on oxygen availability, such as \textit{fnr} mutations (a
global oxygen responsive transcriptional regulator), NO$_2$ (electron
transport decoupler), agitation of the growth medium, actual oxygen levels,
etc. This is expected as the function of \textit{sodA} is linked to 
processes related to oxygen availability.
%% , one can expect its expression level to change prominently under these 34
%% contrasts,
COLOMBOS indeed successfully identifies them as the most relevant contrasts of
\textit{sodA}. 
%
There are also 17 contrasts belong to ontology term `growth', with three in
common with the aforementioned 34 contrasts. The term `growth' is very
general, grouping conditions that trigger various biological processes
simultaneously at a global cellular scale. Hence it is not unexpected to find
\textit{sodA} differentially expressed under these contrasts.
%belonging to this ontology term.




\subsubsection{Extend the genes of a module}

\begin{small} % Step 3

\paragraph{Step 3} Next, we will extend the module with additional genes.
%
\begin{figure}[b]
	\centering
  	\includegraphics[width=1\textwidth]{Fig7-Gene-extending-panel.png}
	\caption[COLOMBOS `ranked gene selection' interface]{
	\textbf{The `ranked gene selection' interface to extend module `sodA\_c3' 
	with the most relevant genes.} 
	The red circle indicates \textit{sodA} in the module. As it's expression 
	profile is perfectly correlated with itself (the correlation score of 1), 
	the circle is located at point (1, 0).}
	\label{fig:colombos-gene-extend}
\end{figure}
%
%To extend the module with additional genes, we will use the edit function. 
Click `edit module' button to go to the `Edit' tab, where three options are
available: edit the name/description, the genes, or the contrasts. Click `edit
genes' to modify the module genes. A gene modification panel appears with all
options to edit genes.  The options in green boxes indicate ways to add genes
to the module, whereas those in red boxes indicate ways to remove genes. 
%% We will not explain all the options here. Interested users please refer the
%% tutorial and/or the help sections of the website for detailed explanations.
%
For this case study, we will use the last option `Add new genes based on
expression' to add additional genes, whose expression patterns are most
(anti-)correlated to the expression pattern of the module.  Click it to bring
up a `ranked gene selection' panel for gene selection.
%% Click the green box surrounding the text of this option; a `ranked gene
%% selection' panel will appear in the `Edit' tab.
\\
This panel (Figure \ref{fig:colombos-gene-extend}) functions in a similar way 
to the `ranked contrast selection' panel explained in Step 1.3. 
%
On the left side, a gene list ranked by the degree of the correlation of each
gene's expression profile with the mean profile of the module genes under the
selected contrasts belonging to this module.  
%
Different types of correlation scores are available to rank the
genes. Interested user can refer Appendix \ref{apd:gene-score} for the
detailed explanation of the different options.  Here, the default one, which
calculates the \textit{Uncentered Pearson Correlation}, is utilized.  The
higher the score, the more similar the profiles are.  Similarly, a cutoff
value can be specified to select only those genes with higher scores.
%% %
%% Based on the \textit{(uncentered) correlation}, there are three different 
%% types of correlation scores available to rank the genes. 
%% Here we will use the default `correlation', which calculates the 
%% \textit{Uncentered Pearson Correlation} as the score. 
%% Other options can be chosen from the dropdown box that will appear by clicking 
%% the text `correlation' in the bar on top of the ranked gene list. 
%% Please refer Appendix \ref{apd:gene-score} for more information 
%% about these options and how to calculate them. 
%
Also in the panel, the figures to the right
% are meant to provide additional helpful information and
show, from left to right, a density plot displaying the distribution of the
correlation scores across all genes in the compendium and a plot showing the
number of genes that will be added if a given cutoff is selected. The circles
connected by a red vertical line in the density plot indicate the gene(s) that
already exists in the current module and its correlation score.

To extend our module, we will use a cutoff value of 0.75.
%% Input 0.75 in the box above the gene list at the left, and click `Ok' button
%% at the side. The list is then filtered, and all genes to be added are
%% automatically selected (checked) in the list to the left.
%
The option `Create new module upon edit' (located below the gene list) is
checked, so that a new module is created with the extended gene set and the
original one is kept unchanged.
%% With the option `Create new module upon edit' (located below the gene
%% list), the original module is kept unchanged, and the extended module is
%% saved as a new module. Chose this option, then
%
When ready, click the `Add' button at left, and specify `sodA\_c3\_g.75' (in
the pop-up window that appears) to name the new module.  A new node
representing newly created module is added below the original module
`sodA\_c3' in the workspace.
%% as the name of the new module to create it. It will now appear as a node
%% below the original module `sodA\_c3' in the workspace.
%
\end{small} % Step 3

The resulting module now contains 35 genes with an expression profile similar
to that of \textit{sodA} under the selected contrasts (see heatmap, Figure
\ref{fig:colombos-heatmap-final}).  Amongst the genes in the module are
\textit{cyoA}, \textit{cyoB}, \textit{cyoC}, \textit{cyoE}, subunits of the
cytochrome b terminal oxidase complex involved in aerobic respiration
\cite{Cotter1992}, \textit{sdhA}, \textit{sdhC}, \textit{sdhD}, encoding the
succinate dehydrogenase (SdhCDAB) active during Krebs cycle catalyzing the
oxidation of succinate to fumarate under aerobic conditions \cite{Wilde1986},
and \textit{sucA}, \textit{sucB}, \textit{sucC}, \textit{sucD} involved in
generating succinyl-CoA, one of the reactants in the Krebs cycle
\cite{Buck1989}.  
%
Considering the functions of these gene products (enriched GO terms in Table
\ref{tab:GOenrich}), it is not at all unexpected that their expression levels
are influenced by oxygen availability.  Moreover, many genes in module
`sodA\_c3\_g.75' also share common regulators with \textit{sodA}, such as
ArcA, Fur, FNR, CRP, etc. Hence, \textit{sodA} (encoding a superoxide
dismutase activity) being coexpressed with the genes involved in aerobic
respiration might be essential to protect the cell against oxidative stress.
%
\begin{figure}[tb]
	\centering
  	\includegraphics[width=1\textwidth]{Fig8-Heatmap-final-module-tuned.png}
	\caption[COLOMBOS heatmap of module `sodA\_c3\_g.75']{
	\textbf{The heatmap of AU2 module `sodA\_c3\_g.75'.}
	The row of expression data that corresponds to \textit{sodA} is
	marked by the arrow. The expression data of the gene groups that are 
	prominently coexpressed with \textit{sodA}, namely cyoABCD, sdhACD and 
	sucBCD, are marked by the named boxes.}
	\label{fig:colombos-heatmap-final}
\end{figure}


\subsubsection{Summary}

The biological relevance between genes and contrasts of module
`sodA\_c3\_g.75' clearly shows the strength of the simple methods implemented
in COLOMBOS.
%
Although the coexpression of module genes does not necessarily imply
coregulation of its genes, 
%
the promoter regions of these genes could be investigated further using motif
detection methods to discover common regulators (see
Figure \ref{fig:workflow-distiller-colombos}).
%% this could be investigated further (see
%% Figure \ref{fig:workflow-distiller-colombos}).  The promoter regions of the
%% genes in this module could be used as input for motif detection to discover
%% common regulators.
%
This task could be achieved either by screening these regions with Position
Specific Scoring Matrices (PSSMs) of previously characterized motifs, or by
using \textit{de novo} motif detection methods \cite{Tompa2005, Storms2010}.
%
Moreover that some genes in the module are unannotated, this shows that
COLOMBOS web service can also serve as a tool to help biologists identify
interesting research candidates.




\subsection{Identifying transcriptional modules with DISTILLER}

DISTILLER \cite{Lemmens2009} is a data integration framework that
automatically identifies condition dependent transcriptional modules by
combining expression data with information on the interaction between a
regulator and its corresponding target genes (here through motif data). Here,
we will first show how to setup the related parameters and run each of the
three steps of DISTILLER through our web service \cite{DISTILLER}.
%
Then after transcriptional modules are obtained,
%
they are visualized together with the motif information in an interactive way
using software ViTraM \cite{Sun2009}.
%
%% the software ViTraM \cite{Sun2009} is used to visualize overlapping modules
%% together with the motif information in an interactive way.
At last, we briefly discuss those resulting modules containing \textit{sodA}.

\subsubsection{Part.1 Running DISTILLER algorithm}\label{sec:dist-distmodule}

DISTILLER automatically identifies all regulatory modules met certain
criterion from the expression and motif data of all genes of the species.
%
The target specific ones can then be filtered out from the resulting modules.
%
In this case study, we will use DISTILLER to identify those modules containing
\textit{sodA}.
%
Consequently, we can simplify the motif data by removing motifs irrelevant to
the query genes.  This greatly reduces the running time of DISTILLER.  After
this simplification, only 7 motifs (ArcA, Rob, SoxS, Fur, IHF, FNR, MarA, and
CRP) remain.


%% Every search for regulatory modules targeting one specific gene or a set of 
%% genes starts from the expression and motif data of all genes of the species. 
%% DISTILLER will then identify all possible modules. From the resulting modules, 
%% the user can filter out those involving his or her query genes. As a case 
%% study, we will use DISTILLER to identify the regulatory modules containing 
%% \textit{sodA}.

%% To reduce the running time of DISTILLER, we simply the motif data by removing
%% motifs that are irrelevant to the query genes.  Here, we omitted those that
%% were not present in the \textit{sodA} promoter region. Only 7 motifs (ArcA,
%% Rob, SoxS, Fur, IHF, FNR, MarA, and CRP) remain.


\paragraph{Create a user profile and a project}
DISTILLER algorithm is computationally expensive, as it explores the complete
search space. Depending on the specified parameters and the dataset size, each
step could take hours or days to finish.  
%
Consequently, a registration is required, which creates an account with
minimally user's email information.  This provides great flexibility, as user
can simply setup the system to run a step of the algorithm and leave the site.
When process finished, user is notified by an email (details in Section
\ref{sec:distiller-email}).  He can then log in and select the corresponding
project to continue.


%% DISTILLER algorithm is computationally expensive, as it explores the complete
%% search space. Depending on the specified parameters and the dataset size, each
%% step could take hours or days to finish.  
%% %
%% Hence, a user account is required
%% with minimally user's email information, and a project should be defined for
%% each dataset.  In this manner, user can simply setup the system to run each
%% step of the algorithm and leave the site.  When process finished, user is
%% notified by an email (see Section \ref{sec:distiller-email} for its format and
%% content).  He can then log in and select the specific project to check the
%% results or continue with next step.

\begin{small} % Step 1,2,3 
%Since DISTILLER can be time consuming, a user account needs to be created and 
%a project should be de?ned for each dataset.

% \vspace{-5mm}\begin{adjustwidth}{3mm}{0cm}
% \nointerlineskip\leavevmode
% command required to use \subparagraph inside adjustwidth env.
% ref: 
%http://tex.stackexchange.com/questions/60992/strange-compilation-problem-using-adjustwidth-environment-from-changepage-packag
\subparagraph{Step 1}	Go to the DISTILLER web service \cite{DISTILLER}. 
% An existing user can login through the front page. 
A new user can be created in three steps, click the `New User' button,
complete the required information, and click the `Create User' button.  The
user's email address is required for sending notification emails (see Section
\ref{sec:distiller-email}).

\subparagraph{Step 2}	After log in, the project selection interface will 
appear, where one can either select an existing project to continue, or create
a new one by clicking `Create New Project' and giving a name and a brief
description for it. Here, we create a new project named `Ecoli\_sodA'.

\subparagraph{Step 3}	After choosing or creating a project, the DISTILLER 
panel will appear where the user can run the three main steps of DISTILLER,
each of which has its own separate tab. For a new project, only the `Module
Detection' tab is accessible (Figure \ref{fig:distiller}), while the `Module
Filtering' and `Module Extension' tabs are disabled.
%\end{adjustwidth}

\end{small} % Step 1,2,3 



\paragraph{Identifying seed modules}

\begin{small} % Step 4,5,6
%\vspace{-5mm}\begin{adjustwidth}{3mm}{0cm}
%\nointerlineskip\leavevmode
\subparagraph{Step 4} For a new project, user need to first upload the required 
expression data file and motif data file (see Section
\ref{sec:distiller-infile-format}) to be analyzed by DISTILLER program, and
preprocess them to check their format and integrity.  As the system supports
expression data file in different formats and compression types, it is user's
responsibility to specify the correct options for the `Compression' and `Data
file format' items in the `Expression Data Parameters' section.
%% information by selecting the right options for the `Compression' and `Data
%% file format' items of the `Expression Data Parameters' respectively.
%
Here, the sample files, expdata\_DISTILLER\_Expression.txt and
binary\_DISTILLER\_Motif.txt, are used.  Upload them using data file specific
upload buttons, specify `Flat file format' as `Data file format' and `No' for
`Compression', then click the `Preprocessing' button to start the process on
the server.
%
As it is time consuming, the user will be notified by email when the process
is done.

%For a new project, no source data files have been uploaded, hence only the
%`Load file' button at the lower right corner is enabled, and the `run
%Distiller' button is disabled (Figure  \ref{fig:distiller}).  Users must first
%upload the expression data file (use sample file
%expdata\_DISTILLER\_Expression.txt) and the motif data file
%(binary\_DISTILLER\_Motif.txt).  Upload the required data files by clicking
%the `Load File' button. Note that the web service automatically recognizes
%the data type of a file according to the prefix of its name.  Please refer
%Note 10 for the mandatory file name prefix of each input file type.

\subparagraph{Step 5} After receiving the notification email, user can go back to 
their project at DISTILLER site.  The interface is the same as seen in Step 4.
Only now the button `Run Distiller' is unlocked.  Three groups of parameters
need to be specified before starting the process: the binary data parameters
for the motif data (see Step 5.1), the expression data parameters (see Step
5.2), and the general ones (see Step 5.3).

%% When receiving the notification email of the preprocessing step, the user can
%% go back to the website to proceed with the main steps.  Open the DISTILLER
%% website, log in with your user account, and select the previously created
%% project. The same interface as seen in Step 4 appears.  There are three groups
%% of parameters need to be specified for this step: the binary data parameters
%% related to the motif data (see Step 5.1), the expression data parameters (see
%% Step 5.2), and the general parameters (see Step 5.3).

\subparagraph{Step 5.1} Binary data parameters: 1) `Binary Support' specifies 
the minimal number of motifs that the genes in a module should have in common;
2) `Binary Thresholds' is the minimal score a motif instance in the promoter
region of a gene should have to be considered as present.  
%
As DISTILLER only accepts binary motif data as input, whereas most motif
detection algorithms generate probabilities scoring the certainty of motif
gene relations, the specified threshold is applied on the motif data to
convert the \textit{continuous values} into the \textit{binary ones}.
%% Any value which is equal or higher than the threshold will be converted
%% into value 1, and 0 otherwise, resulting in a binary matrix.  This binary
%% matrix is then used by the algorithm to identify modules.
%
For this case study, we choose 1 for the Binary Support and 
0.999 as the Binary Threshold. Hence the genes belonging to a module must share 
at least 1 motif and a gene is considered to have a specific motif if the 
corresponding value in the motif data is equal or higher than 0.999.

\subparagraph{Step 5.2}	Expression data parameters: 1) `Box Support' specifies 
the minimal number of condition contrasts under which the genes in a module
should be coexpressed; 2) `Box P-value' is used to generate the threshold
bandwidth sequence that is needed to test whether a module passes the
constraints on the expression data, i.e. whether the genes in the module are
sufficiently coexpressed within the selected contrasts \cite{Lemmens2009}.
Here, 100 are chosen for the `Box Support' and 0.0001 for the `Box P-values'.

\subparagraph{Step 5.3} General parameters: 1) `Minimal Module Size' specifies 
the minimal number of genes that should be in a module; 2) `Number of
Randomizations' specifies the number of random modules that will be generated
for computing a threshold bandwidth sequence for the condition contrast
selection.  Here, we specify 4 and 10000 for these two parameters
respectively.
%% the `Minimal Module Size' as 4, and the `Number of Randomizations' as
%% 10000.
These require that the algorithm generates 10000 random modules consisting of
4 genes to compute the threshold bandwidth sequence.

\subparagraph{Step 6} When all parameters are specified, click `Run 
Distiller'. DISTILLER will now enumerate all initial modules, each of which
contains at least 4 genes coexpressed in more than 100 contrasts and sharing
at least one motif. Note these contrasts are selected based on comparing the
ordered contrasts bandwidth sequences of given genes with the threshold
bandwidth sequence (see \cite{Lemmens2009} for details).
%
When finished, user will receive a notification email (see Section 
\ref{sec:distiller-email}) containing the link to the output file with the
identified modules, and the link to the supplementary file bundle that is 
required to visualize modules using ViTraM.

\end{small} % Step 4,5,6


\paragraph{Filter the raw DISTILLER output} After obtaining the initial modules, 
the user can go back to the saved project at DISTILLER site.  Now, the tabs
`Module Filtering' and the tab `Module Extending' are enabled.  Note that the
user can directly proceed to the module extending step, but due to the
discussion in Section \ref{sec:dist-distiller}, performing a filtering step
before extending the obtained modules is highly recommended.

\begin{small} % Step 7
\subparagraph{Step 7} Click the `Filtering' tab, specify the number of 
modules to be filtered out in this step, then click on `Filter Result' to
proceed.  In our case study, the top 20 modules ranked by their scores
computed by DISTILLER are selected (see Section
\ref{sec:dist-distiller}). Similarly, the result of the process will be
notified by email.
\end{small}

\paragraph{Extend DISTILLER seed modules} After obtaining the filtered results, 
the user can now proceed with the `Module Extension' step.
%% When receiving the notification email for the filtering step, the user can now
%% go back to the DISTILLER website to proceed with the `Module Extension' step.
%
In this step, extra genes are incorporated into a module if they comply with
the relaxed criteria.

\begin{small} % Step 8,9,10

\subparagraph{Step 8} Two parameters need to be specified to run this 
step. Candidate genes have to comply with both parameters in order to be
included in the extended module. The only exception is made for the genes
belong to an \textit{operon} (see Step 9).

\subparagraph{Step 8.1} `Extended Motif Threshold': this is the same type of 
parameter as the `Binary Thresholds' specified in Step 5.1, but less stringent
to allow a gene having more present motifs.
%% By choosing a less stringent (lower) value for the Extended Motif Threshold
%% than was chosen for the Binary Thresholds, a gene will have more motif
%% candidates.
%
As a result, more genes can satisfy the `Binary Support' parameter.  Here, we
use 0.95 for this parameter (as compared to 0.999 for the `Binary
Thresholds').

\subparagraph{Step 8.2}	`Correlation Percentage': a correlation threshold to 
select candidate genes for module extension.  Given a `Correlation Percentage'
of $p$, only those genes whose expression profile correlations, calculated
based on the mean expression profile of a module, higher than $p$, are
considered as the candidates. Here we choose $p$ as 0.95.


\subparagraph{Step 9} Optionally, the genes of a module can be further 
extended with \textit{operon information} if available.  Candidate genes
belonging to an operon, whose first gene is present in a seed module, only
need to satisfy the criterion for the expression profile (`Correlation
Percentage') to be included into the module.  
%
The operon information (i.e. which genes belong to which operon) is included
in the analysis by uploading a file containing the corresponding data (see
Section \ref{sec:distiller-infile-format}).

\subparagraph{Step 10} After specifying the parameters and optionally 
uploading an operon data file, the user can click `Extend Modules' to run the
process.  A notification email is sent when the process is finished.

\end{small} % Step 8,9,10

After receiving the final output file of DISTILLER, the user can then
specifically select the modules that contain his or her query genes, in our
case the gene \textit{sodA}, to continue.

%After receiving the final output file of DISTILLER, the user can then 
%specifically select the modules that contain his or her query genes, in our 
%case the gene sodA. Note that here we only discussed how to use DISTILLER to 
%identify transcription networks. However, a versatile tool like DISTILLER has 
%many other application domains. Please see \textbf{Note 12} for more details. 


\subsubsection{Part.2 Visualization of DISTILLER modules with ViTraM}
The software ViTraM is used to analyse modules discovered by DISTILLER.
However, the DISTILLER output files need to be converted into ViTraM
compatible format first.  Then, ViTraM can be utilized to visualize the
modules.

DISTILLER is non-deterministic due to the randomization process utilized by
the algorithm to select contrasts.
%
Thus, it is possible that when following exactly the same way as outlined here
to run DISTILLER, the recovered modules are different from those obtained in
this example (as provided in `DISTILLER\_outputExtended.m' sample file, see
Section \ref{sec:dist-sample}).  Use the corresponding sample file to proceed
with exactly the same results.


\paragraph{Prepare ViTraM readable files}

\begin{small} % Step 1,2 ViTraM
\subparagraph{Step 1} Download the XMLCreator from the ViTraM 
website\cite{ViTraM} and unzip the file. Under linux, run the
XMLCreator by typing the command `java -jar -Xms256m -Xmx512m
XMLCreator.jar' in a terminal.  Windows users can click file
`XMLCreator\_Windows\_512M.bat' to run it.

\subparagraph{Step 2} In the pop-up window, choose `DISTILLER' as module 
detection tool and click on `OK'.  
%
The main interface appears where user can specify the input and output
files.
%
Two required input files are the DISTILLER `.m' output data file and
the `Expression Data' file (see Section \ref{sec:dist-vitram}).  In
addition, other files providing extra information can be visualized
along with the modules. This includes but not limited to, the motif
information of each gene, the genes' functional annotations, or the
experimental factors and/or sample characters of each , etc.
%
Finally, the names of the output files need to be specified. Two
output files will be generated: the module XML file and the
corresponding expression data file containing the expression values
for those genes and contrasts presented in any of the modules.
%% When ready, click the `Make XML File' button to generate these
%% output files. If files are generated successfully, a window will
%% appear to notify user.


\end{small} % Step 1,2 ViTraM

Here, we demonstrate this step using the corresponding sample files, the
`DISTILLER\_outputExtended.m' and the `expdata\_DISTILLER\_supple.txt'.
%
Additionally, the file `binary\_DISTILLER\_Motif\_supple.txt' providing extra
motif information for each module is specified as the `Motif Data'.
%
We then run XMLCreator to generate the two output files, `ViTraM\_Modules.XML'
and `ViTraM\_Expression.txt'.
%% As an example, we will use sample files provided with this chapter to show how 
%% to run XMLCreator. First, specify the file `DISTILLER\_outputExtended.m' as the 
%% `DISTILLER output data', and the file `expdata\_DISTILLER\_supple.txt' as the 
%% `Expression Data'. This is the minimal input that is required to run 
%% XMLCreator.
%% In addition, we also provide the file `binary\_DISTILLER\_Motif\_supple.txt' as 
%% the `Motif Data' providing extra motif information to be visualized together 
%% with the modules.
%% Specify `ViTraM\_Modules.XML' and `ViTraM\_Expression.txt' as the output file 
%% names in data field `XML Data' and `Expression Data'. Click `Make XML File' 
%% should generate the output file successfully.
%
To run XMLCreator on the DISTILLER output, pleases always use the supplement
files specified in the notification email that accompanies the `.m' file (see
Section \ref{sec:distiller-email}).


\paragraph{Run ViTraM}

From the ViTraM website \cite{ViTraM}, the user can 
download ViTraM v2.0 and unzip the file. Note, a registration is required by 
providing us some basic information, and the user must accept our software 
license.

\begin{small} % Step 3-10 ViTraM

\subparagraph{Step 3} To run ViTraM under windows, click on 
`ViTraM\_Windows\_512M.bat' or `ViTraM\_Windows\_2G.bat' provided.  The latter
can handle larger datasets, but does require more than 2Gb physical memory in
the computer. To run ViTraM under linux or mac, call `ViTraM\_Linux\&Mac.sh'
file.

\subparagraph{Step 4} Load `ViTraM\_Modules.XML' and `ViTraM\_Expression.txt' 
generated in Step 2 by operations `Open Module XML File' and `Open Expression
File' in the 'Input' panel respectively.  After the data are loaded, they can
be visualized by `Load All Modules' in the `Module Selection' panel.  This
extra loading step enables visualizing a subset of modules, where directly
visualizing all the modules in a large dataset generally causes memory issues.
In such a case, user can use `Filter' panel first to select only the relevant
modules, then visualize them. This significantly reduces the loading time to
visualize modules.

\subparagraph{Step 5} Subsequently click on `View Modules' in the 
`Module Display' panel to show the modules in the main window.  In this window,
the genes are listed at the left and the conditions at the top, and each module
is represented as one or a group of boxes with its id shown at the top left
corner of each box and a distinctive color for each module.
% Note, each module uses a distinctive color for its boxes.

\subparagraph{Step 6} Click `View Gene Properties' in `Gene Properties 
Display' panel, the extra properties of the gene can be visualized together
with the modules in an extra window located at the left side of the `Module
Display' window. 
%
The motif information provided in Step 2 will now appear here.
%% Since we provided motif data while generating XML file in Step 2 of this
%% section, they will now appear in this window.
%
The color of each cubic represents the value of the motif score explained in
Section \ref{sec:dist-vitram}.  The red color corresponds to the high value and
the green the low one. If a score is higher than a threshold, a cross will
appear in the corresponding cubic.
%% For more information about this panel, we refer user to the manual of
%% ViTraM.

%% \subparagraph{Step 7} The `View Experiment Properties' in the `Experiment 
%% Properties Display' panel can open up one extra window that shows the relevant
%% condition data if loaded. In this case study, no such data is available.

\subparagraph{Step 7} By clicking on `Favorite Genes' in the `Filter' panel 
at the right hand side, it is possible to display only those modules containing
a particular gene.
%
The pop-up window shows on the left a list of all genes. Select the gene
`b3908' (\textit{sodA}), and click the arrow button `\textrightarrow' to add it
to the favorite gene list at the right.  Then click `OK'.  Only modules
containing \textit{sodA} will still be in display.

\subparagraph{Step 8} To optimize the visualization of modules that overlap, 
click on `Automatic ordering' in the `Module ordering' panel and subsequently
`Run Overlap Index' to layout the currently displayed modules in the most
optimal way (for details on the algorithms that identify the optimal display of
overlapping modules, we refer to \cite{Sun2009}).

\subparagraph{Step 9} Click on `View Modules' and subsequently `Refresh 
Modules' in the `Overview \& Heatmap Display'. Then click on `Adding Heatmap'. 
The expression values of the genes for the condition contrasts in the currently 
displayed modules will be shown by means of a heatmap.

\end{small} % Step 3-10 ViTraM


\subsubsection{Analyzing resulting DISTILLER modules containing \textit{sodA}}

In this case study, we used DISTILLER to search for modules of coexpressed
genes sharing at least one motif instances for the same regulator.  Resulting
modules including the motif instances shared among genes within each module are
listed in Table \ref{tab:distModules}.  Since we are interested in
\textit{sodA}, transcriptional modules 2, 3, 8, and 12 containing this gene
were selected and visualized using ViTraM (Figure \ref{fig:vitram}).  
%
The numbering of the modules corresponds to their ranks in the filtered
results. Note that as a global method, DISTILLER identifies all possible
regulatory modules in the dataset (not only those with \textit{sodA}). So the
numbering of the modules containing \textit{sodA} is not consecutive. The lower
the number, the more significant a module is.

\begin{figure}[b]
	\centering
  	\includegraphics[width=1\textwidth]{Fig9-Distiller-modules-color.pdf}
	\caption[DISTILLER \textit{sodA} related modules visualized in ViTraM] 
	{\textbf{DISTILLER \textit{sodA} related modules visualized in ViTraM.}
	The figure shows the four modules containing \textit{sodA}. Each module is 
	indicated with one or more squares and expression values are indicated by 
%	heatmap. Module 2 is indicated by dotted lines, module 3 by solid line, 
%	module 8 by dash-dotted lines, and module 12 by dashed lines.}
	heatmap. Module 2 is indicated by blue lines, module 3 by red line, 
	module 8 by green lines, and module 12 by yellow lines.}
	\label{fig:vitram}
\end{figure}

\begin{table}[tb]
	\caption{Overview of the 20 transcriptional modules identified by DISTILLER}
	\label{tab:distModules}
	\begin{small}
	\begin{tabular}{p{5mm} p{2cm} c c p{4.5cm}}
	\toprule
	{\bf ID} & {\bf Motifs} & {\bf \# Genes} & {\bf \# Conds} & {\bf Genes in 
	\textit{sodA} modules} \\
	\midrule
	1 & Fur	& 22 & 97 & \\
	2 &	MarA, SoxS & 4 & 77	& \textit{fpr}, \textit{poxB}, 	
	\textit{\textbf{sodA}}, \textit{zwf} \\
	3 &	ArcA, CRP &	7 &	79 & \textit{acnA}, \textit{acnB}, \textit{aldA}, 
	\textit{gltA}, \textit{osmY}, \textit{sdhC}, \textit{\textbf{sodA}} \\
	4 &	FNR, IHF & 4 & 226 & \\
	5 &	ArcA, FNR &	4 &	123	& \\	
	6 &	CRP	& 308 &	107	& \\
	7 &	CRP, FNR & 4 & 87 & \\
	8 &	Fur, CRP & 4 & 146 & \textit{cyoA}, \textit{nupC}, \textit{sdhC}, 
	\textit{\textbf{sodA}} \\
	9 &	SoxS & 4 &	181	& \\
	10 & Fur & 18 & 86	& \\
	11 & MarA &	4 &	206	& \\
	12 & CRP, FNR & 5 &	90 & \textit{aldA}, \textit{cyoA}, 
	\textit{malP}, \textit{pdhR}, \textit{\textbf{sodA}} \\
	13 & IHF &	8 &	87	& \\
	14 & IHF &	53 & 123	& \\
	15 & FNR &	38 & 102	& \\
	16 & CRP &	76 & 125	& \\
	17 & ArcA, CRP &  4 &	148	& \\
	18 & SoxS &	4 &	95	& \\
	19 & Fur & 4 &	100	& \\
	20 & Fur & 22 &	76 & \\
	\bottomrule
	\end{tabular}
	\end{small}
\end{table}

\begin{table}[tb]
\begin{threeparttable}
	\caption{GO enrichment of \textit{sodA} modules\tnote{1}}
	\label{tab:GOenrich}
	\begin{small}
	\begin{tabular}{p{2.2cm} c c p{5.3cm}}
	\toprule
	{\bf Module} & {\bf P-value} & {\bf GO ID} & {\bf GO Term Description} \\
	\midrule
	{\bf\it COLOMBOS} \\
	\hline
	\textit{sodA\_c3\_g.75}	& 2.41e-12 & GO:0006091 & generation of precursor 
	metabolites and energy \\
	& 9.23e-02 & GO:0006800 & oxygen and reactive oxygen species metabolic 
		process \\
	& 3.17e-02 & GO:0006869 & lipid transport \\
	& 2.74e-07&	GO:0022900&	electron transport chain \\
	& 4.22e-02&	GO:0042592&	homeostatic process \\	
	& 1.85e-02&	GO:0045454&	cell redox homeostasis \\[2ex]
	{\bf\it DISTILLER} \\
	\hline
	\textit{module2} & 7.07e-02&	GO:0005996&	monosaccharide metabolic 
	process \\
	& 7.11e-03&	GO:0006800&	oxygen and reactive oxygen species metabolic 
		process \\
	\hline
	\textit{module3} & 3.11e-05&	GO:0006091& generation of precursor 
	metabolites 
	and energy \\
	& 1.77e-02&	GO:0006800&	oxygen and reactive oxygen species metabolic 
	process \\
	& 7.11e-02&	GO:0044262&	cellular carbohydrate metabolic process \\
	\hline
	\textit{module8} & 4.09e-03&	GO:0006091&	generation of precursor 
	metabolites 
	and energy \\
	& 1.06e-02&	GO:0006800&	oxygen and reactive oxygen species metabolic 
	process \\
	& 4.09e-03&	GO:0022900&	electron transport chain \\
	\hline
	\textit{module12}& 1.42e-02& GO:0006800& oxygen and reactive oxygen 
	species metabolic process \\
	\bottomrule
	\end{tabular}
	\begin{tablenotes}
	\item[1] Enriched GO terms are ordered by GO access ID.
	\end{tablenotes}
	\end{small}
\end{threeparttable}
\end{table}


Module 2 contains the genes regulated by both SoxS and MarA. The identified
contrasts in this module belong to the ontology terms `carbohydrate metabolic
process', `growth', and `response to oxidative stress'.
%
SoxS and MarA are known to participate in the removal of superoxide and nitric
oxide and protection from organic solvents \cite{Touati2000}.  One can expect
that the contrasts belonging to `response to oxidative stress' to be suitable
for inducing their expression and therefore the expression of genes regulated
by them. 
%
We already highlighted the link between the removal of superoxide species and
metabolic pathways.  It explains why the condition contrasts under
`carbohydrate metabolic process' are found in this module.  
%
As discussed after Step 3 of Section \ref{sec:dist-module-colombos}, to observe
condition contrasts belonging to `growth' is not unreasonable in this case.  
%
In addition to \textit{sodA}, the other genes found in this module are
\textit{fpr} (Ferredoxin-NADP reductase), \textit{poxB} (Pyruvate
dehydrogenase), and \textit{zwf} (Glucose-6-phosphate 1-dehydrogenase). The
enriched GO terms of this gene set (Table \ref{tab:GOenrich}) are highly
consistent with the observed condition ontology terms.


In module 3, genes are regulated by both ArcA and CRP. Its condition
contrasts are highly related to the ontology terms `growth' and
`response to oxygen levels'. 
%
CRP is a global regulator involves in the degradation of any non-glucose carbon
sources and also an antagonist of catabolite repression. 
%
On the other hand, ArcA participates specifically in a signal transduction
system sensing particular aerobic and anaerobic growth conditions
\cite{Compan1994}.  
%
The functions of the genes found in this module (\textit{acnA}, \textit{acnB},
\textit{aldA}, \textit{gltA}, \textit{osmY}, \textit{sdhC}, and \textit{sodA})
are an intersection between global metabolic pathways and specific processes
responding to oxygen level changes (see Table \ref{tab:GOenrich}).


Genes in module 8 are both regulated by CRP and Fur. As previously
mentioned, CRP is a global regulator that facilitates bacterial fitness
in function of the availability of different carbon sources. 
%
Fur is a sensor of intercellular iron concentration, and also participates in
the response to reactive nitrogen species \cite{Mukhopadhyay2004}.  
%
The contrasts of this module mainly involve `carbohydrate metabolic process',
`growth', `detoxification of nitrogen compound', `lactose catabolic
process'. 
%
Except for \textit{sodA}, the genes found in this module are \textit{cyoA}
(Ubiquinol oxidase subunit 2), \textit{nupC} (Nucleoside permease nupC), and
\textit{sdhC} (Succinate dehydrogenase cytochrome b$_{556}$ subunit).
%
Altogether the condition contrasts and the genes selected in this module
reflect the linkage between basic metabolism changes induced by external
C-source availability (e.g. glucose concentration) and intra-cellular energy
production through the electron transfer chain (see also enriched GO terms in
Table \ref{tab:GOenrich}).


In module 12, genes are regulated by CRP and FNR, both master-regulators. 
%
FNR activates genes involved in anaerobic metabolism and represses genes
involved in aerobic metabolism.  The enriched GO term (only `oxygen and
reactive oxygen species metabolic process') for this module's gene set reflect
this.
%
The selected contrasts of this module constitute a very broad set of ontology
terms, including `carbohydrate metabolic process', `growth', `dna repair', `sos
response', `lactose catabolic process', and `detoxification of nitrogen
compound'.  This is in line with the global regulation exerted by CRP and FNR.



\section{Discussion and Conclusion}

The COLOMBOS web service provides very straightforward and intuitive
ways for analyzing and exploring large-scale expression compendia. It is
intended for specific queries based on users pre-knowledge on genes or
conditions of interest. In each step, this knowledge and user
involvement is crucial to explore various functionalities in order to
generate good quality results. 
%
It is a deterministic approach, which means that when executing exactly the
same operations, COLOMBOS will generate exactly the same result. Furthermore,
the analysis methods implemented in COLOMBOS are fast, generally taking only
seconds or less to generate results for each step.

DISTILLER, on the other hand, is a semi-automatic method. Except for a limited
set of parameters, it requires very little user involvement. Guided by both
motif information and expression data, the method tries to compose a global
regulatory network that covers the whole input dataset (see Table
\ref{tab:distModules}).
%
It is non-deterministic due to the randomization involved in generating a
threshold bandwidth sequence for the condition selection (see Step 8 in Section
\ref{sec:dist-distmodule}).

In this case study, we have used COLOMBOS and DISTILLER to gain insights in the
functional processes, in which \textit{sodA} is involved, and the regulatory
programs that coordinate them. This resulted in one module identified by
COLOMBOS and four modules by DISTILLER.  As discussed in previous
sections, each module does reflect relevant biological processes, in which
genes including \textit{sodA} participate.  This does not necessarily require
the modules to overlap, as each might represent different processes involving
different genes.  When these two approaches are used to address the same
biological question as was done here, the distinct features of each approach
lead to different but meaningful and thus complementary results.

The module `sodA\_c3\_g.75' identified by COLOMBOS contains several genes
involved in various biological processes (see
Figure \ref{fig:colombos-heatmap-m1} and the discussion at the end of Section
\ref{sec:dist-module-colombos}) with very high and coherent expression values
under most contrasts (in both cases where up-regulation or down-regulation
occurred). In contrast, when compared to module `sodA\_c3\_g.75', each of the
DISTILLER modules contains in general much less genes, which show less extreme
expression levels indicating a weaker coexpression signal compared to
`sodA\_c3\_g.75'.

When evaluating the overlap between modules found by these two methods, module
3 and `sodA\_c3\_g.75' share four genes (\textit{aldA}, \textit{acnB},
\textit{sdhC}, \textit{sodA}) and 19 contrasts (`response to oxygen level',
`growth').  These contrasts represent approximately 25\% and 35\% of all
condition contrasts of each module respectively. Module 8 and `sodA\_c3\_g.75'
have three genes (\textit{cyoA}, \textit{sdhC}, \textit{sodA}) and eight
condition contrasts (`response to oxygen level') in common, which represent
approximately 6\% and 15\% of all condition contrasts in each module.  This
kind of similarity can be expected, since we applied both methods to answer
same biological question based on the same expression data.

On the other hand, no overlap is observed between module 2 and
`sodA\_c3\_g.75'.  Module 12 shares three genes (\textit{aldA}, \textit{cyoA},
\textit{sodA}) with `sodA\_c3\_g.75' though, but the two modules have no
contrasts in common.  Upon closer inspection of the expression data, genes in
DISTILLER module 2 are coexpressed, but the expression value of \textit{sodA}
is not very high for these contrasts, when compared with those of
`sodA\_c3\_g.75'. Hence those contrasts will not be easily picked up by
COLOMBOS in Step 2.3 of Section \ref{sec:dist-module-colombos} (unless a much
lower selection threshold is used).  Furthermore, genes appearing only in the
DISTILLER modules show a different expression behavior from that of
\textit{sodA} under those contrasts selected by COLOMBOS.  As a result, when
extending genes of a module based on expression profiles (see Step 4 of Section
\ref{sec:dist-module-colombos}), those genes will be ranked less
relevant. Consequently, they will hardly be considered as prime candidates for
module extension.

In summary, COLOMBOS focuses on using expression values as its main criterion
for building modules, resulting in clear expression changes in tight
coexpression. It can easily extract prominent coexpression behaviour in the
expression data, but might miss modules with less significant coexpression
patterns. The coexpression patterns retrieved by COLOMBOS can be broadly
interpreted as genes being functionally related as their abundance are altered
in similar ways in response to various stimuli. This functional relationship
could imply coregulation (which might be identified using motif detection
algorithms, see also Figure \ref{fig:workflow-distiller-colombos}), but it is not
a necessary prerequisite.
Instead, it cannot be excluded that several regulatory programs might be 
responsible for the observed coexpression patterns.
On the other hand, DISTILLER, as a global method, tries to recover
distinctive modules that can be directly linked to a shared regulatory
program, i.e. of which the genes are coregulated by the same (set of)
regulator(s). Combining motif information with expression data, it
successfully retrieves less prominent coexpression patterns with
biological significance by utilizing extra information regarding the
presence (or prediction, depending on the nature of the motif input
data) of transcription factor binding sites in its genes promoter
regions. The case study of \textit{sodA} presented in this chapter illustrates 
this complementarity of these two approaches for retrieving biologically 
relevant results. 

In this chapter, we focus on the specific application of the COLOMBOS and
DISTILLER.  However, versatile tools as them have many other functionalities
and application domains.
%
In COLOMBOS, various options exist to build the module based on the information
other than a given set of genes.  Furthermore, if desirable, the module data
can be downloaded for further analysis.  As an example, the data of the module
`sodA\_c3\_g.75' is exported into file
`expdata\_COLOMBOS\_module\_information.txt' (available as a sample file, see
Section \ref{sec:dist-sample}).
%
DISTILLER can use other data sources as input, for instance, the information 
on the binding of regulators as obtained from ChIP-chip experiments 
\cite{Lemmens2009, Lemmens2006}.
Furthermore, it can be used together with data source other than motif data to 
discover other type of networks, as long as connections exist between the data 
source and the gene expression profile. 
Two such example data sets are protein-protein interaction data or synthetic
lethality data.



%Here we only briefly described the functionalities of COLOMBOS related to 
%constructing modules based only on a known gene set. However, various options 
%exist to build the module based on different types of information. It is also 
%possible to download the data of a module to further analyze 
%it (see Appendix \ref{apd:data-download}). As an example, the data of the 
%module `sodA\_c3\_g.75' obtained in here is exported into file 
%`expdata\_COLOMBOS\_module\_information.txt' (available 
%as a sample file, see Section \ref{sec:dist-sample}). 
%Interested users can refer to Appendix \ref{apd:colombos} for more 
%information. 
%% %
%Here we only discussed how to use DISTILLER to identify transcription 
%networks. However, a versatile tool like DISTILLER has many other application 
%domains. 
%It can use other data sources as input, for instance, the information 
%on the binding of regulators as obtained from ChIP-chip experiments 
%\cite{Lemmens2009, Lemmens2006}.
%Furthermore, it can be used together with data source other than motif data to 
%discover other type of networks, as long as connections exist between the data 
%source and the gene expression profile. 
%For instance, protein-protein interaction data or synthetic lethality data can 
%be included. 
%Binary data is preferred; for other types of data, the preprocessing step can 
%convert them into binary ones.

%%%%%%%%%%%%%%%%%%%%%%%%%%%%%%%%%%%%%%%%%%%%%%%%%%
% Keep the following \cleardoublepage at the end of this file, 
% otherwise \includeonly includes empty pages.
\cleardoublepage
 
% vim: tw=70 nocindent expandtab foldmethod=marker foldmarker={{{}{,}{}}}
