\chapter{Conclusions and Perspectives}\label{ch:conclusion}

\instructionsconclusions



\section{Summary and achievements}

Chapter \ref{ch:command} illustrated a novel methodology to create 
organism-specific cross-platform expression compendium.  
%
The uniqueness of the methodology lies in the capability of integrating
expression data across different platforms.
% by resolving data collection issue and data inconsistency.
%
Special attention has been paid on two aspects, resolving data representation
heterogeneity, particularly those related to the data generated on dual-channel
microarrays, and improving data consistency and compatibility.
%
The method has two advantages over the single platform approach.  First, it
facilitates the creation of a compendium that incorporates far more data than
existing one, providing a more comprehensive gene expression landscape for a
species.  Second, it enables the construction of a sizable compendium for
species without dominant microarray platform.
%
Moreover, a web system named COMMAND is developed, providing user friendly
interfaces and guidance to facilitate compendium creation and maintenance.
%
The system is the only known one that is capable of partially automating the
tedious microarray expression data retrieval task, enabling the handling of a
large volume of publicly available data.
%
The utility of the methodology has been proven by successfully creating first
three (\cite{Engelen2011}) and later on four extra bacteria compendia
(\cite{Meysman2014}), and one eukaryotic compendium for monocot \textit{Zea
  mays} (\cite{Fu2014}).



% from COLOMBOS abstract
Chapter \ref{ch:colombos} presents three comprehensive organism-specific 
cross-platform expression compendia for the bacterial model organisms 
(\textit{Escherichia coli}, \textit{Bacillus subtilis}, and \textit{Salmonella 
enterica} serovar Typhimurium), and the accompany web access portal COLOMBOS.
%
Each compendium also incorporates extensive annotations for both genes and 
experimental conditions; these heterogeneous data are integrated 
in the COLOMBOS analysis tools to interactively browse and query the 
compendia. 
%not only for specific genes or experiments, but also metabolic pathways, 
%transcriptional regulation mechanisms, experimental conditions, biological 
%processes, etc.
%
The web portal has been directly utilized in various studies, to identify 
additional gene coexpressed with targets (\cite{Meysman2011, Fu2012, 
Meysman2014a}), to identify conditions under which genes of 
interests are coexpressed (\cite{Desai2013}).  
%
The compendia, which can be downloaded in entirety and studied 
utilizing different system biology approaches (\cite{Lemmens2009, Michoel2009, 
Zhao2011, DeSmet2011, Zarrineh2011}), have been incorporated in diverse 
researches, to construct co-expression network (\cite{Cloots2011, Kolar2012}), 
to reconstruct transcriptional regulatory network (\cite{Faria2013}), to 
understand the physiological mechanism driving the response to the 
environmental changes (\cite{Balderas-Martinez2013}), and to study expression 
conservation and divergence between species (\cite{Meysman2013}).
%
%The formalized condition contrast annotation found in
%COLOMBOS has made it ideal for linking gene expression
%changes to the underlying causal factors, such as activa-
%tion of transcription regulators by effectors (15)or
%genomic mutations (16)
%
% Check colombos 2 paper for relevant applications and references



% from Abstract
Chapter \ref{ch:distiller} discusses how to discover condition 
dependent co-expression modules containing specific query genes in expression 
compendium utilizing two complementary methods COLOMBOS and DISTILLER.
%
The former is designed for query-driven interactive data explorations in 
expression compendium alone, whereas the latter generates a global regulatory 
network overview through integrating expression data with extra evidence, e.g. 
motif data, in a unsupervised fashion.
%
Both methods generate biologically relevant however distinctive modules for the 
query gene \textit{sodA}.
%
The case study demonstrates that COLOMBOS is best to extract prominent 
coexpression behavior among functionally related genes, whereas DISTILLER, 
guided with the motif information, recovers co-regulated genes albeit with a 
less prominent coexpression patterns.
%
Their applications hence are driven by the type of the biological question 
asked.
%
The case study will surely alleviate the difficulty faced by biologist to 
choose and utilize these powerful methods.



% from MAGIC abstract
Chapter \ref{ch:magic} describes an expression compendium for \textit{Zea mays} 
integrating large amount of publicly available data (1749 microarrays in 69 
experiments over 27 platforms). 
%
Uniquely, the probe sequences of all 27 platforms included are obtained, and a 
complete probe re-annotation based on the \textit{Zea mays} 5b.60 genome 
release 
is constructed.  Incorporating this re-annotation to build the compendium 
greatly improves the consistency between the data of different platform origin.
%
Additionally, the condition annotation system is revamped to reflect the 
complex lifestyle of plant, specifying not only the external perturbations 
at the contrast level, but also the internal sample attributes, including 
genotype (breeding line), tissue, and development stage.
%
This compendium is made available through an upgraded web portal MAGIC that 
hosts a variety of analysis tools utilizing the extended annotations for 
easy data browsing and analysis.
%
The uniqueness and the high quality of maize compendium coupled with the 
friendly system to explore it will surely make this a valuable resource for 
biologist studying this species.


In summary, the main contributions of the research work presented in this 
thesis are twofold.
%
First, we developed a unique data integration methodology to create 
cross-platform expression compendium from publicly available data, and 
developed a system to facilitate the creation of such an compendium.
%
Second, we have created such compendia for several species and developed web 
portals to serve them to the community.  They have been proven and will 
continue to be valuable resources for the researcher who study those or 
relevant species.





% various issues with data and annotation, exploration









\section{Future work}



\subsubsection*{Compendium creation and curation}


%% incorporating the latest advances in technology and computational method to
%% improve data consistency and quality, and updating with the newly generated
%% datasets to extend the scope of the compendia.

The first and foremost task is to keep the existing compendia up-to-date, by
updating them with the newly generated datasets and expending the annotation to 
reflect the ever broadening scope of the data.
%
New revision of the existing compendia will be updated every half a year
incorporating additional experiments.
%
When the genome annotation of the corresponding species is revised, a new
release will be generated.
%
Upon the public release of COLOMBOS v2.0 \cite{Meysman2014}, the existing
compendia for bacteria model organism \textit{E. coli}, \textit{B. subtilis},
and \textit{S. enterica} serovar Typhimurium has been extensively expanded, and
four new compendia for bacteria \textit{S. coelicolor} \textit{P. aeruginosa}
\textit{M. tuberculosis} \textit{H. pylori} have been added.



% \cite{Shendure2008}  earliest NGS review
% \cite{Nagalakshmi2010}  
% \cite{Ozsolak2011}   RNAseq review 
% \cite{Croucher2010a,Mader2011}  RNAseq for bacteria
% 
% RNAseq advantage -> RNAseq algorithm less mature -> RNAseq microarray compatible -> include RNAseq data in our compendium  \cite{Taminau2012}
%
%% RNA-seq is a sequencing-based methods utilizing the advanced next generation
%% sequencing (NGS) technology.  In contrast to the hybridization based method,
%% such as microarray, the actual sequences of cDNA or RNA is directly determined
%% by the platform, whereas the relative abundance of an RNA molecular is derived
%% from the quantity of the mapped reads.


Based on novel next generation sequencing (NGS) technology, RNA sequencing
(RNA-seq) quantifies relative genes expression level by directly sequencing the
expressed transcripts then estimating the abundance from the generated reads.
%
In contrast to the hybridization based microarray, such a sequence-based method
does not rely on pre-existing knowledge of gene sequences, instead is capable
of determining the actual sequence of RNA and quantifying expression level for
individual isoforms of genes.
%
Moreover, it has been shown that RNA-seq data have higher sensitivity, less
variation, and broader dynamic range than microarray ones
\cite{Sirbu2012,Marioni2008}.
% which opens up the gate for many applications, such as, gene annotation,
% novel isoform discovery.
%
Due to its superiority and swiftly dropping application cost, RNA-seq is
quickly replacing microarray to become the standard method for gene expression
studies.
%
It is then crucial that our system should evolve to incorporate this type of
data.
%
A straightforward approach has been implemented in COLOMBOS v2.0
\cite{Meysman2014}, in which the RNA-seq data are first mapped to genome, then
genes relative expression levels are estimated from the mapped reads, and at 
last, log-ratios are calculated based on predefined sample pairs.
%
The validity of the approach has been shown using real experiment data
\cite{Meysman2014}.
%
The approach has enabled a swift incorporation of RNA-seq data into the
existing compendia, albeit with a restrained scope.
%, as large amount of information contained in such data is ignored.
%
Further research is needed to revise the existing database model to include
other details that are omitted by current approach, such as RNA sequence
information, transcript level expression, etc.
%
Novel data exploration, analysis, and visualization methods need to be
developed to handle expression data of different granularities under a common
framework.
%% \textbf{ATLAS2013}
%% %
%% Visualization: focusing on genome browser coverage views of
%% expression allowing the user to observe in detail how expression is
%% distributed across different exons and transcripts of a given gene.
%
Moreover, both NGS and RNA-seq are still the young and rapidly changing
fields. We will continuously evaluate new computational methods upon available
and incorporate them when appropriate to improve the data quality.




% Carolina
%
%% - RNA-seq data for transcriptomics: The system should evolve to include these 
%% type of data. This means to revisit the protocols of data submition for RNAseq 
%% and update the preprocessing steps. The database model is flexible to include 
%% other details that are usually omitted on microarray data: expression levels at 
%% the transcript level and not only gene level (I mean a gene with multiple 
%% alternative transcripts). With RNA-seq data it is possible to have transcript 
%% specific data as well, but the gene level values should remain in the system to 
%% allow comparison to microarray data.

% Single-cell RNA-seq (single cell level)
%
% Refs:
% 
% http://nextgenseek.com/2014/01/paper-summary-single-cell-rna-seq-reveals-dynamic-random-monoallelic-gene-expression/
% http://nextgenseek.com/2014/03/multiple-papers-on-random-monoallelic-expression-by-rna-seq/



The automation of various tasks for raw data collection often involves
straightforward keywords matching and pattern recognition, the rich free text
descriptive information are ignored by the current system.
%
The lack of training set has prevented us utilizing text mining technologies to
analyze this information.
%
Now, successfully parsed data provide an ample training set to make this
feasible.
%
It is then desirable to employ advanced text mining technology to explore the
rich descriptive information to supplement existing methods in hopes of
improving the overall performance of the system.
%
The same idea can be employed to develop system that assists manual annotation
curation by suggesting relevant properties through analyzing meta data
retrieved from the online repositories.
%
Similar system ZOOMA \cite{ZOOMA} has been developed by EBI for such purpose.
However, it maps only onto Experimental Factor Ontology (EFO) term used by EBI
databases.
%
%% % ONTOLOGY exists but not well adoped -> require manual curation -> system to
%% % improve efficiency
%% \cite{Jamieson2012}
%% %
%% Text mining technology has been successfully utilized in various similar
%% efforts \cite{Kapushesky2010} `ZOOMA'.
%% % 
%% % \cite{DeBodt2012} did not find in the text ...
%% % \cite{Zinman2013} same keyword and pattern matching methods as us (see supplementary methods)







%% \textbf{Aggregation}
%% Aggregation: methods to aggregate expression changes of a gene under a
%% given category of experimental condition variations \textbf{ATLAS2013}
%% %
%% Aggregated type of expression analysis, such as those available in
%% Genevestigator (3 genevestigator papers, v3 section 4)
%% %
%% Pay attention that in Genevestigator, certain analysis is done on intensity 
%% data (directly integrable across chips), and our data are log-ratios !!
%
%% \textbf{gene set} analysis and expression aggregation \textbf{ATLAS2013}



%% \textbf{more compendium}
%
%% ENCODE and other project about other expressed elements of genome
%% % Refs: http://nextgenseek.com/2013/10/large-scale-sequencing-efforts-go-beyond-genetic-variations-towards-function/
%
%% - Expand to human/mouse ??
%
%% Eukaryotic expression compendium (in general, not platform specific)
%% - biallelic, diallelic
%% - multiple alleism
%
%% (at populational level)



%% \textbf{data sharing and publicity} BioMart, DAS, etc





\subsubsection*{Compendium exploration and visualization}

As exemplified in this research work, a cross-platform compendium provides
comprehensive expression profiles not only for different gene also for a
variety of environmental and genetic variations (contrast).
% which incrementally approximates the `true' profile with each incorporation
% of new data. % asymptote
One interesting feature would be allowing users to review their specific study
from the respective of existing knowledge.
%
A straightforward approach could be allowing user to upload their own data and
compare it with what available in the compendium to identify contrasts
possessing similar expression variations.
%
Such an analysis has been exemplified in the early compendium paper
\cite{Hughes2000}.
%
Alternatively, relevant experiments selected based on annotations could be
visualized together with their own work to investigate diverse mechanisms
underlying a common phenotype.
%
%% Allowing incorporating user own data set, and comparsion \cite{DeBodt2010}.




%% Higher level of visualization (more interaction)
%% - pathway centric  (Systems Biology Markup Language (SBML))
%% - integrating with Regulatory network, showing relevant sub-network connected to DEGs
%
Initially, the web access portal only allows data to be visualized as (possibly
overlapping) heatmap.  The function to visualize genes and the corresponding
functional annotations as an interactive network for individual module has been
added in COLOMBOS v2.0 \cite{Meysman2014}.
%
Data visualization for system biology is a booming field, in which a variety of
tools with different focus exists \cite{Gehlenborg2010}.
%
It will be interesting to explore this rich resource to either incorporate
advanced methods into our system or link out to external services to provide
better visualization and improve the data interpretability.
%
%% The review of Gehlenborg \textit{et. al.} \cite{Gehlenborg2010} has provided
%% many interensting leads to follow.
%
%% To study the dynamics of an network by generating modules under variant 
%% conditions to probe the network model, to discover core and peripheral 
%% structures involved, etc ... (literature) 




%% Carolina
%
%% - Expand the capabilities of colombos to query genes. For instance, the system 
%% could integrate for instance more interactive queries with known molecular 
%% pathways, so the users could follow a certain number of pathways or 
%% protein-protein interactions across multiple conditions.




%% Carolina
%
%% Adapt the system (COMMAND and COLOMBOS) to other types of omics data, such
%% as miRNA or metabolomics data.
























\section{Perspective}


%% A global transcription landscape (not only mRNAs)

%% % RNAseq for smallRNA and microRNA
%% %
%% % http://seqanswers.com/forums/showthread.php?t=19680
%% If you are truly interested in longer non coding you will need Total
%% RNA-seq libraries (deplete rRNAs and tRNAs).  For anything smaller you
%% need to do the small RNA protocol for things under ~50 bp.
%% %
%% having all three, Total RNA (long ncRNA), mRNA, and small (smallRNA,
%% ncRNA, microRNA), while more work and cost is incredibly informative.

%% miRNA precursor ?

%% \textbf{microRNA} The pipeline used to process RNA-sequencing data for
%% Expression Atlas will be enhanced to analyse microRNA RNA-sequencing
%% experiments. Subsequently, good quality microRNA RNA-sequencing
%% experiments available in ArrayExpress will be re-processed and
%% included in Expression Atlas \textbf{ATLAS2013}.
%% %
%% \textbf{smallRNA}
%% %
%% As siRNA and miRNA uses different protocols, different bioinformatics
%% might be needed. (Reference?)

Except for the messenger RNA which ultimately translated into protein, there are
many other RNA molecules that are transcribed in a cell.
%
They are commonly referred as non-coding RNA (ncRNA), as they do not code for
protein.
%
Among them, transfer RNAs and ribosome RNAs have been well studied, as they
carry out crucial biological functions in a cell and are well conserved across
the tree of life.
%
In the last decades, the functionality of other ncRNAs has gradually been
revealed.
%
Micro RNA (miRNA) induced silencing has been identified as one of the major
post-transcriptional regulation mechanism that is well preserved among
eukaryotes\cite{Chen2007a}.
%
Whereas, long non-coding RNAs (lncRNAs) are found to take up many regulatory
roles, being transcriptional, post-transcriptional, or even
epigenetic\cite{Baker2011}.
%
Other types of functioning ncRNAs have also been discovered, including, small
regulatory RNA (sRNA), small inference RNA (siRNA), small nuclear RNA (snRNA),
etc.
%
As the specific adaptations of common technologies, such as, tiling array and
RNA-seq, for ncRNA detection become mature, more and more ncRNA expression data
will be generated.
%
Our compendium creation methodology could well be adapted to include ncRNAs
expression information into mRNA compendium or even construct ncRNA specific
compendium, which can be utilized to study their expression pattern and
investigate their regulatory roles in connection with environmental changes.
%
One such modification is to recruit specialized preprocessing
procedures developed to handle miRNA specific microarray platforms
(e.g., Affymetrix miRNA GeneChips and Agilent miRNA microarrays) to
improve data quality and consistency.
%
As a proof of concept, by identifying sRNA targeting probes through platform 
reannotation, a special \textit{E. coli} compendium containing only data 
generated on E. coli Antisense Genome Array and E. coli Genome 2.0 Array has 
been created to study sRNA-target interactions and to extend known regulatory 
network with post-transcriptional networks \cite{Ishchukov2014}.




















%% across species

Comparative genomics is a field in which the genomic features of different
organism are compared to study the evolutionary mechanisms and the phylogeny
among organisms.  Such study has enabled knowledge transfer between species and
facilitated gene annotation and regulatory elements identification.
%
Traditionally, such study has focused primarily on analyzing sequence-based
features, including gene sequences, gene order (synteny, genetic linkage),
regulatory sequences, protein sequences, protein domains, etc.
%
Integrating functional genomics information, such as, expression data, in the
comparative study has been shown to provide new insights for study
conservation and divergence \cite{Bergmann2004}.
%
The availability of comprehensive compendia for multiple bacterial species in
COLOMBOS has facilitated research that studies expression conservation and
divergence between \textit{E. coli} and \textit{S. enterica} sev. Typhimurium
at global level \cite{Meysman2013}.
%
Alternatively, algorithms, such as COMODO \cite{Zarrineh2011} and cMonkey
\cite{Waltman2010}, do exist that can simultaneously explore heterogeneous
expression data sets of multiple species to identify biclusters containing
conserved core orthologous genes.
%
The mehodology presented in this thesis will enable the creation of such
comprehensive compendia for many species, which can benefit the comparative
genomics study of various scopes.
%
Moreover, it will be very interesting to develop web systems that
facilitate such analysis using existing algorithms, and novel methods
to analyze and visualize the results obtained.
%
%% Cross-species analysis and visualizatoin, paralogues and homeologues, - COMMODO
%% \cite{Zarrineh2011} - multi-species bicluster (cMonkey)
%% \cite{Waltman2010,Kacmarczyk2011}




%% Across different molecular level (RNA, protein, metabolites, etc)

%% Integrating different data sets
%% \cite{Troyanskaya2005,DeKeersmaecker2006,Joyce2006} [carolina, conclusion]

%% % carolina, comments for conclusion
%% Systems biology and data integration: You could introduce the human
%% case and the TCGA or the PAN-cancer projects: these initiatives
%% provide measurements at multiple molecular level, such as mRNA, miRNA,
%% methylation, proteomics, CNV and somatic mutations. Experiments that
%% measure multiple levels for the same samples are also becoming
%% available for other organisms such as mouse, or even plants (mRNA,
%% metabolites, etc..)  The challenge is to explore these data levels
%% simultaneously to derive more reliable hypothesis about the molecular
%% changes affecting individuals.... (you can complement with more
%% blabla)

%% % \cite{Fierro2008} conclusion
%% %
%% With the increasing number of high throughput technolo- gies, we can
%% expect that compendia for other ``omics data'' will also grow at an
%% increasing pace. Each compendium provides a snap shot of the
%% condition-dependent changes at a certain cellular level. 
%% %
%% A huge challenge remains of how a comprehensive view of the cellular
%% machinery can be built by combining all these individual snap shots
%% [73-75]. 
%% %
%% An important and often overlooked issue with the meta-analysis of
%% biological data is the context-dependency, the condition dependency of
%% the interactions, their timing, and their location.

A cell is a complex and dynamic system whose phenotypes are driven by
the interplay between molecules at different layers, including, DNA,
RNA, protein, metabolites, etc.
%
Global profiles of molecules of single-layer and/or certain character
of them only provide incomplete observations for certain aspects of
such a system, and often contaminated with specific biological or
technical bias.
%
Combining knowledge of different aspects in research to provide a
comprehensive and integrated view of cellular machinery, however, can
greatly reduce the influence of such bias and allow more precise
system level models to be built.
%
%% For example, The Cancer Genome Atlas (TCGA) research network has
%% systematically collected molecular profiles at different levels for
%% many tumor types in order to improve our ability to diagnose, treat,
%% and prevent cancer through a better understanding of the genetic basis
%% of this disease.
%
Such an integrative approach has been successfully utilized 
%% to develop novel cis-regulatory module prediction algorithms with
%% increased precision \cite{Herrmann2012, Sun2012},
to better characterize consequences of genetic variations of
individual cancer and compare such an integrated profile across
different types of cancers in order to sift out causal mutations
\cite{Weinstein2013}, and to generate integrative personal omics
profile (iPOP) providing comprehensive information for personalized
disease diagnosis and treatment \cite{Chen2012}.
%
%% A huge challenge remains to study cell as a system
%% \cite{Ideker2001,Chuang2010} throught combining knowledge of different
%% apsects to build a comprehensive view of cellular machinery and to
%% study the phenotype driven mechanism at system level.
%% %
Compendium, created using the method developed here, aiming to provide
a complete landscape for one type of profiles of cell by combining
results of individual studies will remain as a premium data source to
support this kind of research.
%
It is of great interests to utilize the insights we gained through
creating expression compendia to develop methodologies that builds
other types compendia, and to develop new computational methods and
service that allows an integrated analysis across different types
compendia of certain species.





%% EFO is not specific enough, covering too many different things ... 

%% Incorporating Experimental Factor Ontology (EFO) (created and used by 
%% ArrayEpress ATLAS) 
%% %% ref: http://www.ebi.ac.uk/efo/






%%%%%%%%%%%%%%%%%%%%%%%%%%%%%%%%%%%%%%%%%%%%%%%%%%
% Keep the following \cleardoublepage at the end of this file, 
% otherwise \includeonly includes empty pages.
\cleardoublepage

% vim: tw=70 nocindent expandtab foldmethod=marker foldmarker={{{}{,}{}}}
