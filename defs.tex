% Some useful packages
\usepackage{amssymb,amsthm}
\usepackage{amsmath}

% useful: TODO notes
\usepackage{todonotes}
%fix placement of todo notes (see todonotes manual)
\setlength{\marginparwidth}{(\stockwidth-\columnwidth)/2-.35cm}

% Some nice colors
\newcommand{\red}[1]{\textcolor{red}{#1}}
\newcommand{\green}[1]{\textcolor{green}{#1}}
\newcommand{\blue}[1]{\textcolor{blue}{#1}}
\newcommand{\cyan}[1]{\textcolor{cyan}{#1}}
\newcommand{\magenta}[1]{\textcolor{magenta}{#1}}
\newcommand{\yellow}[1]{\textcolor{yellow}{#1}}
\newcommand{\orange}[1]{\textcolor{orange}{#1}}
\newcommand{\violet}[1]{\textcolor{violet}{#1}}
\newcommand{\purple}[1]{\textcolor{purple}{#1}}
\newcommand{\brown}[1]{\textcolor{brown}{#1}}
\newcommand{\gray}[1]{\textcolor{gray}{#1}}
\newcommand{\darkgray}[1]{\textcolor{darkgray}{#1}}
\newcommand{\lightgray}[1]{\textcolor{lightgray}{#1}}
\newcommand{\bred}[1]{\textbf{\textcolor{red}{#1}}}


% Personal settings [FU]


\renewcommand{\arraystretch}{1.2}  % use 1.2 line space in table




\usepackage[table]{xcolor} % use grey color for even rows in table
\usepackage{textcomp} % for straight quote sign
\usepackage{threeparttable} % for table with extra notes
\usepackage{threeparttablex} % extend threeparttable to work with longtable 
%package
\usepackage{array} % redefined tabular env. for table
\usepackage{rotating} 
% NOTE: if pdfsync is required, use with novbox option, otherwise mess up with tabular env.
% \usepackage[novbox]{pdfsync}
% Ref: http://tex.stackexchange.com/questions/49487/tabular-tabularx-automatic-linebreak-p-m-b-doesnt-work-as-expected
\usepackage{booktabs} % to provide hlines of different weight (thickness)
% Ref: 
%http://tex.stackexchange.com/questions/3445/latex-tables-how-do-i-make-bold-horizontal-lines-typically-hline
\usepackage{colortbl}	% grey line

\usepackage{footmisc} % for reuse footnote using label


\usepackage[font={small,it}]{caption} % overwrite default by using small and 
%italic font for caption
\usepackage[Q=yes,pverb-linebreak=no]{examplep}
% introduce \Q{} command to input underscore as normal text
% ref: http://tex.stackexchange.com/questions/48632/underscores-in-words-text
\usepackage{upgreek} % for up stand greek characters, e.g. \upmu

\usepackage{changepage} % to adjust row adjustwidth for manual text indentation
% Ref: http://tex.stackexchange.com/questions/35933/indenting-a-whole-paragraph/

\usepackage{hyperref}

\usepackage{setspace} % to adjust line space


% NOTE: this section has been commented out, as the issue with bib file is solved externally, check bib-thesis.readme

%% Unicode char support
%
%
%\usepackage[utf8]{inputenc} % for input coding
%\usepackage[T1]{fontenc} % for output, what fonts to use for printing characters.
%\usepackage{lmodern}
%%\usepackage[utf8x]{inputenc} % [utf8x] option has more extensive coverage but is not as well supported
%%\usepackage{CJKutf8}
%%
%% NOTE: refer http://tex.stackexchange.com/questions/44694/fontenc-vs-inputenc


% Adjust the top margin and text height to reduce pages. But this only reduces  
% 4 out of 124 pages, not worth the efforts
% 
%\addtolength{\topmargin}{-.10in}
%\addtolength{\textheight}{.5in}